\documentclass{article}
\usepackage{amsmath}
\usepackage{xcolor}
\usepackage{amsthm}
\usepackage{graphicx}
\usepackage{hyperref}
\usepackage{datetime}
\usepackage{todonotes}

% inline comments
% you could use \listoftodos to print an overview
\newcommand{\inline}[1]{ {\color{blue}{#1}}\addcontentsline{tdo}{todo}{#1}}
\newcommand{\comment}[1]{{\color{blue}\noindent{#1}\\}\addcontentsline{tdo}{todo}{#1}}
% use this one to disable
%\newcommand{\inline}[1]{\ignorespaces}


\newdateformat{monthyeardate}{\monthname[\THEMONTH] \THEYEAR}

\newcommand{\newmarkedtheorem}[1]{%
  \newenvironment{#1}
    {\pushQED{\qed}\csname inner@#1\endcsname}
    {\popQED\csname endinner@#1\endcsname}%
  \newtheorem{inner@#1}%
}

\theoremstyle{definition}
%\newtheorem{eg}{Example}[section]
\newmarkedtheorem{eg}{Example}[section]
\newtheorem{observation}{Observation}[section]
\theoremstyle{plain}
\newtheorem{define}{Definition}[section]
\newtheorem{proposition}{Proposition}
\newtheorem{theorem}{Theorem}[section]
\newtheorem{assump}{Assumption}[section]
\newtheorem{remark}{Remark}[section]


\title{Network with Finite Buffers}
\author{Jeroen van Riel}
\date{\monthyeardate\today}

\begin{document}

\section*{Bounded lane capacity}

Up to this point, we have not taken into account the fact that lanes between
intersection have finite capacity. We need to incorporate this aspect in order
to develop a model that could be used for practical applications. Under high
traffic loads, lanes with finite buffer capacity can give rise to
\textit{blocking} of upstream intersections. Therefore, the traffic controller
needs to take into account these additional dynamics.


Recall the single intersection scheduling model
\begin{align*}
  \min_{y} \quad & \sum_{i \in \mathcal{N}} y_{i} & \\
  \text{s.t.} \quad & r_{i} \leq y_{i} & \text{ for } i \in \mathcal{N} , \\
  & y_{i} + \rho_{i} \leq y_{j} & \text{ for } (i,j) \in \mathcal{C} , \\
  & y_{i} + \sigma_{i} \leq y_{j} \text{ or } y_{j} + \sigma_{j} \leq y_{i} & \text{ for } \{i,j\} \in \mathcal{D} ,
\end{align*}
where we had
\begin{align*}
  \mathcal{D} &= \{ (i,j) \in \mathcal{N} : l(i) \neq l(j) \} , \\
  \mathcal{C} &= \{ (i,j) \in \mathcal{N} : l(i) = l(j) , \, k(i) + 1 = k(j) \} .
\end{align*}

We start the simplest extension of the single intersection model by considering
two intersections in tandem.
%
We define a graph $(V,E)$ with \textit{labeled edges} as follows.
Let $V$ denote the indices of the intersections.
Let $E$ denote the set of ordered triples $(l, v, w)$ for each lane $l$ whose
vehicles travel from intersection $v$ to $w$.
%
Let $d(v, w)$ denote the minimum time necessary to travel between intersections $v$ and $w$.
Let $b(v, w)$ denote the maximum number of vehicles that can be on the lane between intersections $v$ and $w$.
Let $\mathcal{N}(l)$ denote all the vehicles starting on lane $l$.
Let $v_{0}(i)$ denote the first intersection that vehicle $i$ encounters on its route.
Let $\mathcal{R}(l)$ denote the set of intersections visited by vehicles from lane $l$.
Let $y(i,v)$ denote the crossing time of vehicle $i$ at intersection $v \in V$.
Let $\mathcal{C}^{v}$ and $\mathcal{D}^{v}$ denote the disjunctive and respectively disjunctive pairs for intersection $v \in V$.

Writing $\texttt{conj(\dots)}$ and $\texttt{disj}(\dots)$ for the usual conjunctive and disjunctive constraints, we obtain the following formulation
\begin{subequations}\label{eq:network_problem}
\begin{align}
  \min_{y} \quad & \sum_{i \in \mathcal{N}} \sum_{v \in \mathcal{R}(l(i))} y(i,v) & \\
  \text{s.t.} \quad & r_{i} \leq y(i, v_{0}(i)) & \text{ for } i \in \mathcal{N} , \\
  & \texttt{conj}(y(i,v), y(j,v)) & \text{ for } (i,j) \in \mathcal{C}^{v}, v \in V , \\
  & \texttt{disj}(y(i,v), y(j,v)) & \text{ for } \{i,j\} \in \mathcal{D}^{v}, v \in V , \\
  & y(i, v) + d(v, w) \leq y(i, w) & \text{ for } i \in \mathcal{N}(l), (l, v, w) \in E, \\
  & y(i, w) + \hat{\rho}_{i} \leq y(j, v) & \text{ for } (i,j,v,w) \in \mathcal{F} ,
\end{align}
\end{subequations}
where $\mathcal{F}$ is defined as follows
\begin{align*}
  \mathcal{F} = \{ (i,j,v,w) : i,j \in \mathcal{N}(l), k(i) + b(v,w) + 1 = k(j),  (l,v,w) \in E\} .
\end{align*}
Each $(i,j,v,w) \in \mathcal{F}$ represents a pair of vehicles driving on the
same lane, for which the first vehicle must have left lane segment $(v,w)$
before vehicle $j$ can enter.
%
The constraints involving $\mathcal{F}$ yield the following property of schedules.

\begin{proposition}
  Let $y$ be a solution to the network scheduling
  problem~\eqref{eq:network_problem}. Each lane segment $(l,v,w) \in E$ contains
  never more than $b(v,w)$ vehicles.
\end{proposition}
\begin{proof}
  Let $i$ be a vehicle that has $(v,w)$ on its route. Define the occupancy
  interval $D_{i} = [y(i,v), \, y(i,w)]$, then we say that $i$ occupies $(v,w)$
  at some time $t$ whenever $t \in D_{i}$.
  %
  Therefore, the number of vehicles in $(v,w)$ at time $t$ equals the number of such
  intervals containing $t$.

  Suppose at some time $t$, there are more than $b(v,w)$ vehicles $i$ such that $t \in D_{i}$.
  %
  Let $i_{0}$ be such that $y(i_{0},v) \leq y(i,v)$ for all $i$ such that $t \in D_{i}$.
  By the conjunctive constraints at $v$, we have
  \begin{align*}
    y(i_{0}, v) + \rho_{i_{0}} \leq y(i_{1}, v) + \rho_{i_{1}} \leq \dots \leq y(i_{k}, v) + \rho_{i_{k}} ,
  \end{align*}
  where $i_{k} = (l(i_{0}), k(i_{0}) + k)$.
  {\color{blue} To be continued...}
  %We show that $j=(l(i), k(i) + b(v,w) + 1)$
\end{proof}



% \bibliography{references}
% \bibliographystyle{ieeetr}

\end{document}
