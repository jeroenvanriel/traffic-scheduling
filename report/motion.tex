\documentclass[a4paper]{article}
\usepackage{amsmath}
\usepackage{amssymb}
\usepackage{xcolor}
\usepackage{amsthm}
\usepackage{dsfont}
\usepackage{graphicx}
\usepackage{hyperref}
\usepackage{datetime}
\usepackage{outlines}
\usepackage[round]{natbib}   % omit 'round' option if you prefer square brackets

\usepackage{matlab-prettifier}

\newdateformat{monthyeardate}{\monthname[\THEMONTH] \THEYEAR}

\newcommand{\newmarkedtheorem}[1]{%
  \newenvironment{#1}
    {\pushQED{\qed}\csname inner@#1\endcsname}
    {\popQED\csname endinner@#1\endcsname}%
  \newtheorem{inner@#1}%
}

\theoremstyle{definition}
%\newtheorem{eg}{Example}[section]
\newmarkedtheorem{eg}{Example}[section]
\newtheorem{observation}{Observation}[section]
\newtheorem{define}{Definition}[section]
\theoremstyle{plain}
\newtheorem{proposition}{Proposition}[section]
\newtheorem{theorem}{Theorem}[section]
\newtheorem{assump}{Assumption}[section]
\newtheorem{remark}{Remark}[section]

\author{Jeroen van Riel}
\date{\monthyeardate\today}
\title{Motion Planning}

\begin{document}

\section*{MotionSynthesize}

\begin{align*}
  \texttt{MotionSynthesize}(z_{i,k}(t'_{0}), t'_{0}, t'_{f}, y) := \\
  \text{arg min}_{x : [t'_{0}, t'_{f}] \rightarrow \mathbb{R}} \; &\int_{t_{0}}^{t_{f}} |x(t)|dt \\
  \text{ subject to } \; & \ddot{x}(t) = u(t) , \text{ for all } t \in [t'_{0}, t'_{f}] ; \\
  & 0 \leq \dot{x}(t) \leq v_{m} , \text{ for all } t \in [t'_{0}, t'_{f}] ; \\
  & |u(t)| \leq a_{m} , \text{ for all } t \in [t'_{0}, t'_{f}] ; \\
  & |x(t) - y(t)| \geq l , \text{ for all } t \in [t'_{0}, t'_{f}] ; \\
  & x(t'_{0}) = x_{i,k}(t'_{0}); \quad \dot{x}(t'_{0}) = \dot{x}_{i,k}(t'_{0}) ; \\
  & x(t'_{f}) = 0; \quad \dot{x}(t'_{f}) = v_{m} ,
\end{align*}
%
where initial state $z_{i,k}(t'_{0}) = (x_{i,k}(t'_{0}), \dot{x}_{i,k}(t'_{0}))$.

\subsection*{AMPL implementation}

Using the AMPL modeling language, we can almost immediately implement the above
linear program such that it can be read by a modern solver.


\subsection*{MATLAB implementation}

We start by expressing $v = {(v_{1}, \dots, v_{N})}^{T}$ in terms of the decision variables $u = {(u_{0}, \dots, u_{N-1})}^{T}$.
From the initial condition $v_{0} = v_{m}$ and the relation $v_{i+1} = v_{i} + u_{i} \cdot \Delta t$, we obtain
\begin{align*}
  v = v_{m} \mathds{1} + Au ,
\end{align*}
with the lower triangular matrix
\begin{align*}
  A =
  \begin{pmatrix}
    \Delta t & 0  \\
    \Delta t & \Delta t & 0 &  \\
    \vdots & \ddots & \ddots & \ddots
    \end{pmatrix} ,
\end{align*}
which can be constructed using the following Matlab code:
\begin{lstlisting}[style=Matlab-editor]
  A = tril(delta_t * ones(N))
\end{lstlisting}

Similarly, we express $x = {(x_{1}, \dots, x_{N})}^{T}$ in terms of $v$.
From $x_{0} = -L$ and the relation $x_{i+1} = x_{i} + (v_{i} + v_{i+1}) \cdot \Delta t / 2$, we obtain
\begin{align*}
  x = -L \mathds{1} + Bv ,
\end{align*}
with the matrix
\begin{align*}
  B = \Delta t / 2 \cdot
  \begin{pmatrix}
    1 & 1 \\
    1 & 2 & 1 \\
    1 & 2 & 2 & 1 \\
    \vdots & \vdots & \vdots & \; & \ddots
  \end{pmatrix} ,
\end{align*}
which can be constructed using the following Matlab code:
\begin{lstlisting}[style=Matlab-editor]
  B = delta_t / 2 * (tril(2 * ones(N), 1) - diag(ones(N - 1, 1), 1) - [ ones(N, 1) zeros(N, N-1) ])
\end{lstlisting}

\subsubsection*{Constraints}

We can now start constructing the matrix $C$ and the righ-hand side $b$.

The acceleration constraints
\begin{align*}
  -a_{m} \leq u_{i} \leq a_{m}
\end{align*}
can simply be encoded as
\begin{align*}
  C_{1} = I, \; b_{1} = a_{m} \mathds{1} , \\
  C_{2} = -I, \; b_{2} = a_{m} \mathds{1} .
\end{align*}

The constraints on the velocities
\begin{align*}
  0 \leq v_{i} \leq v_{m}
\end{align*}
are encoded as
\begin{align*}
  C_{3} = A, \; b_{3} = 0 , \\
  C_{4} = -A, \; b_{4} = v_{m} \mathds{1} .
\end{align*}


In order to encode constraint $x_{N} = 0$, observe that
\begin{align*}
  x &= -L \mathds{1} + Bv \\
    &= -L \mathds{1} + B(v_{m} \mathds{1} + Au) , \\
    &= -L \mathds{1} +  v_{m} B \mathds{1} + BA u .
\end{align*}
Let $M[N]$ denote the $N$-th row of matrix $M$, then
\begin{align*}
  x_{N} = -L + v_{m}(B \mathds{1})[N] + (BA)[N]u ,
\end{align*}
so the constraint is encoded as
\begin{align*}
  C_{5} = (BA)[N], \; b_{5} = L - v_{m} (B \mathds{1})[N] , \\
  C_{6} = -(BA)[N], \; b_{6} = -L + v_{m} (B \mathds{1})[N] .
\end{align*}


In order to encode constraint $v_{N} = v_{m}$, observe that
\begin{align*}
  v_{N} = v_{m} + A[N] u ,
\end{align*}
so the constrain is encoded as
\begin{align*}
  C_{7} = A[N], \; b_{7} = 0 , \\
  C_{8} = -A[N], \; b_{8} = 0 .
\end{align*}

The constraints for keeping a safe distance to the vehicle ahead, given by
\begin{align*}
  x_{i} \leq y(t_{0} + i \cdot \Delta t) - l , \; \text{ for all } i
\end{align*}

\subsubsection*{Objective}

Finally, the objective is encoded as
\begin{align*}
  \max_{u} \sum_{i=0}^{N} x_{i} &= \max_{u} \mathds{1}^{T} x \\
  &= \max_{u} \mathds{1}^{T} (-L \mathds{1} + v_{m}B \mathds{1}  + BAu) \\
  &= \max_{u} \mathds{1}^{T} BA u .
\end{align*}

\end{document}
