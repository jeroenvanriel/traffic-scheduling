\documentclass[a4paper]{article}
\usepackage[margin=3.5cm]{geometry}
\usepackage{amsmath}
\usepackage{amssymb}
\usepackage[svgnames]{xcolor}
\usepackage{amsthm}
\makeatletter
\def\th@plain{%
  \thm@notefont{}% same as heading font
  \itshape % body font
}
\def\th@definition{%
  \thm@notefont{}% same as heading font
  \normalfont % body font
}
\makeatother
\usepackage{dsfont}
\usepackage{graphicx}
\usepackage{caption}
\usepackage{hyperref}
\usepackage{datetime}
\usepackage{outlines}
\usepackage{float}
\usepackage{booktabs}
\usepackage{enumitem}
\usepackage{mathtools}
\usepackage{nicematrix}
\usepackage{nccmath}
\usepackage{lipsum}
\usepackage[activate={true,nocompatibility},final,tracking=true,kerning=true,spacing=true,factor=500,stretch=15,shrink=15]{microtype}

\addtolength{\skip\footins}{2mm}

\input{data/knitr_init.tex}

% code highlighting
\usepackage{minted}
\usepackage{xpatch}
\newminted[cminted]{python}{fontsize=\small}
\xpretocmd{\cminted}{\RecustomVerbatimEnvironment{Verbatim}{BVerbatim}{}}{}{}

% link coloring
\hypersetup{
   colorlinks,
   linkcolor={red!90!black},
   citecolor={green!40!black},
   urlcolor={blue!60!black}
}

% concatenation symbol (c.f. ++ in Haskell)
\newcommand\mdoubleplus{\mathbin{+\mkern-10mu+}}

% end of proof symbol
\newcommand{\newmarkedtheorem}[1]{%
  \newenvironment{#1}
    {\pushQED{\qed}\csname inner@#1\endcsname}
    {\popQED\csname endinner@#1\endcsname}%
  \newtheorem{inner@#1}%
}
% \renewenvironment{proof}{{\noindent\bfseries Proof.}}{*something*}
%\let\oldproofname=\proofname
%\renewcommand{\proofname}{\rm\bf{\oldproofname}}


\theoremstyle{definition}
%\newtheorem{eg}{Example}[section]
\newmarkedtheorem{eg}{Example}[section]
\newtheorem{remark}{Remark}
\theoremstyle{plain}
\newtheorem{define}{Definition\hspace{0.25em}\ignorespaces}
\newtheorem{property}{Property\hspace{0.25em}\ignorespaces}
\newtheorem{observation}{Observation}
\newtheorem{proposition}{Proposition}
\newtheorem{lemma}{Lemma\hspace{0.25em}\ignorespaces}
\newtheorem{corollary}{Corollary}
\newtheorem{theorem}{Theorem\hspace{0.25em}\ignorespaces}
\newtheorem{assump}{Assumption\hspace{0.25em}\ignorespaces}

\DeclareMathOperator{\interior}{int}

\newdateformat{monthyeardate}{\monthname[\THEMONTH] \THEYEAR}

\author{Jeroen van Riel} \date{\monthyeardate\today} \title{Vehicle trajectories
  in a tandem of intersections}

\begin{document}

\maketitle

\newcommand\halfopen[2]{\ensuremath{[#1,#2)}}
\newcommand\openhalf[2]{\ensuremath{(#1,#2]}}

\renewcommand{\labelitemii}{\textbullet}
\renewcommand{\labelitemiii}{\textbullet}


Let $\mathcal{D}[a,b]$ denote the set of continuously differentiable functions
$\gamma : [a,b] \rightarrow \mathbb{R}$ satisfying the constraints $0 \leq \dot{\gamma}(t) \leq 1$ and
$-\omega \leq \ddot{\gamma}(t) \leq \omega$ for all $t \in [a,b]$.
%
For $\gamma_{1} \in \mathcal{D}[a_{1}, b_{1}], \gamma_{2} \in \mathcal{D}[a_{2}, b_{2}]$, when
we write $\gamma_{1} \leq \gamma_{2}$ without explicitly mentioning where it applies, we mean
$t \in [a_{1}, b_{1}] \cap [a_{2}, b_{2}]$. We also write $\gamma \leq \min \{ \gamma_{1}, \gamma_{2} \}$ as a shorthand for
$\gamma \leq \gamma_{1}$ and $\gamma \leq \gamma_{2}$.

Given some $\gamma \in \mathcal{D}[a,b]$ and some time $\xi \in [a, b]$, consider the
\emph{stopping trajectory} $\gamma[\xi]$ that is identical to the original trajectory
until $\xi$, from which it starts decelerating to a full stop, so that at time
$t \geq \xi$, the position is given by
\begin{subequations}
\begin{align}
  \gamma[\xi](t) &= \gamma(\xi) + \int_{\xi}^{t} \max\{0, \dot{\gamma}(\xi) - \omega(\tau-\xi) \} d\tau \\
                     &= \gamma(\xi) + \begin{cases}
                                        \dot{\gamma}(\xi)(t-\xi) - \omega(t-\xi)^{2} / 2 &\text{ for } t \leq \xi + \dot{\gamma}(\xi) / \omega , \\
                                        {(\dot{\gamma}(\xi))}^{2} / (2 \omega) &\text{ for } t \geq \xi + \dot{\gamma}(\xi) / \omega .
                                        \end{cases} \label{eq:underbound}
\end{align}
\end{subequations}
This definition guarantees $\gamma[\xi] \in \mathcal{D}\halfopen{a}{\infty}$.
%
Note that a stopping trajectory serves as a lower bound in the sense that, for any
$\mu\in\mathcal{D}[c,d]$ such that $\gamma = \mu$ on $[a, \xi] \cap [c, d]$, we
have $\gamma \leq \mu$ and $\dot{\gamma} \leq \dot{\mu}$.
%
Furthermore, $\gamma[\xi](t)$ is a non-decreasing function in terms of either of its
arguments, while fixing the other. To see this for $\xi$, fix any $t$ and
consider $\xi_{1} \leq \xi_{2}$, then note that $\gamma[\xi_{1}](t)$ is a lower
bound for $\gamma[\xi_{2}](t)$.
%
\begin{property}
  Both $\gamma[\xi](t)$ and $\dot{\gamma}[\xi](t)$ are continuous when considered as functions
  of $(\xi, t)$.
\end{property}
\begin{proof}
  Write $f(\xi, t) := \gamma[\xi](t)$ to emphasize that we are dealing with two variables.
  Recall that $\dot{\gamma}$ is continuous by assumption, so the equation
  $\tau = \xi + \dot{\gamma}(\xi)/\omega$ defines a separation boundary of the
  domain of $f$. Both cases of~\eqref{eq:underbound} are continuous and they
  agree at this boundary, so $f$ is continuous on all of its domain.
  %
  Since $x \mapsto \max\{0, x \}$ is continuous, it is easy to see that also
  $(\xi, t) \mapsto \dot{\gamma}[\xi](t) = \max\{0, \dot{\gamma}(\xi) - \omega (\tau - \xi) \}$
  is continuous.
\end{proof}

Because $\gamma[\xi](t)$ is continuous and non-decreasing in $\xi$, the set
$X(t_{0}, x_{0}) := \{\xi : \gamma[\xi](t_{0}) = x_{0}\}$ is a closed interval
(Lemma~\ref{lemma:levelset}).
%
Define the closed region
$\bar{U} := \{ (t, x) : \gamma[a](t) \leq x \leq \gamma[b](t) \}$. For each
$(t_{0}, x_{0}) \in \bar{U}$, there must be some $\xi_{0}$ such that
$\gamma[\xi_{0}](t_{0}) = x_{0}$, as a consequence of the intermediate value theorem
and the above property.
%
We use $U$ to denote $\bar{U}$ without the points on $\gamma$, by defining
$U := \bar{U} \setminus \{ (t,x) : \gamma(t) = x\}$. Next, we prove that $\gamma[\xi_{0}]$ is actually
unique if $(t_{0}, x_{0}) \in U$, so we may choose
$\xi(t_{0}, x_{0}) := \max X(t_{0}, x_{0})$ as the canonical
representation of this unique trajectory $\gamma[\xi(t_{0}, x_{0})]$.

\begin{property}\label{prop:xi-unique}
  For $(t_{0}, x_{0}) \in U$, if
  $\gamma[\xi_{1}](t_{0}) = \gamma[\xi_{2}](t_{0}) = x_{0}$, then
  $\gamma[\xi_{1}] = \gamma[\xi_{2}]$.
\end{property}
\begin{proof}
  Suppose $t_{0} < \xi_{i}$, then $x_{0} = \gamma[\xi_{i}](t_{0}) = \gamma(t_{0})$ contradicts the assumption $(t_{0}, x_{0}) \in U$.
  %
  Therefore, assume $\xi_{1} \leq \xi_{2} < t_{0}$, without loss of generality.
  %
  Since $\gamma[\xi_{1}] = \gamma[\xi_{2}]$ on $[a, \xi_{1}]$, note that
  we have the lower bounds $\gamma[\xi_{1}] \leq \gamma[\xi_{2}]$ and
  $\dot{\gamma}[\xi_{1}] \leq \dot{\gamma}[\xi_{2}]$.
  %
  We must have $\dot{\gamma}[\xi_{1}](t_{0}) = \dot{\gamma}[\xi_{2}](t_{0})$,
  because otherwise $\gamma[\xi_{1}] > \gamma[\xi_{2}]$ somewhere in a
  sufficiently small neighborhood of $t_{0}$, which contradicts the first lower
  bound.

  Since $\dot{\gamma}[\xi_{1}](\xi_{1}) \leq \dot{\gamma}[\xi_{2}](\xi_{2})$, it
  is clear that $\ddot{\gamma}[\xi_{1}](t) \geq \ddot{\gamma}[\xi_{2}](t)$, for
  $t \geq \xi_{2}$. This implies that
  $\dot{\gamma}[\xi_{1}](t) \geq \dot{\gamma_{2}}(t)$ for $t \geq t_{0}$. This
  in turn implies that $\dot{\gamma}[\xi_{1}](t) = \dot{\gamma}[\xi_{2}](t)$ and
  thus $\gamma[\xi_{1}](t) = \gamma[\xi_{2}](t)$ for $t \geq t_{0}$.

  It remains to show that $\gamma[\xi_{1}] = \gamma[\xi_{2}]$ on $[\xi_{1}, t_{0}]$, so consider the smallest
  $t^{*} \in (\xi_{1}, t_{0})$ such that $\gamma[\xi_{2}](t^{*}) > \gamma[\xi_{1}](t^{*})$.
  %
  Since $\dot{\gamma}[\xi_{1}] \leq \dot{\gamma}[\xi_{2}]$, this implies that $\gamma[\xi_{2}](t) > \gamma[\xi_{1}](t)$ for
  all $t \geq t^{*}$, but this contradicts our assumption that
  $\gamma[\xi_{2}](t_{0}) = \gamma[\xi_{1}](t_{0})$.
\end{proof}

% previous argument of the above property
%
% Suppose we have $\xi_{1} \leq \xi_{2}$ such that $\gamma[\xi_{1}](\tau) = \gamma[\xi_{2}](\tau)$
% for some $\tau \geq \xi_{1}$, then we must have
% $\dot{\gamma}[\xi_{1}](\tau) = \dot{\gamma}[\xi_{2}](\tau)$, because otherwise
% $\gamma[\xi_{1}]$ and $\gamma[\xi_{2}]$ would be crossing at time $\tau$, which
% contradicts the fact that $\gamma[\xi_{1}]$ is a lower bound for $\gamma[\xi_{2}]$.
% Hence, $\gamma[\xi_{1}] = \gamma[\xi_{2}]$ for the whole domain
% $\halfopen{a}{\infty}$.
% %
% Such a situation occurs whenever $\xi_{1}$ and $\xi_{2}$ both lie in an
% interval where $\ddot{\gamma}(t) = -\omega$.


\begin{lemma}\label{lemma:curvejoining}
  Let $\gamma_{1} \in \mathcal{D}[a_{1}, b_{1}]$ and $\gamma_{2} \in \mathcal{D}[a_{2}, b_{2}]$ be
  intersecting at exactly one time $t_{c}$ and assume
  $\dot{\gamma}_{1}(t_{c}) > \dot{\gamma}_{2}(t_{c})$, then under the conditions
  \begin{enumerate}[label=(\roman*)\quad,leftmargin=5em]
    \item $\gamma_{2} \geq \gamma_{1}[a_{1}]$,
    \item $b_{2} \geq t_{f} := t_{c} + \dot{\gamma}_{1}(t_{c}) / \omega$,
  \end{enumerate}
  there is a unique trajectory $\psi$ such that
  \begin{enumerate}[label=\textbullet\quad,leftmargin=5em]
    \item $\psi = \gamma_{1}[\xi]$, for some $\xi < t_{c}$,
    \item $\psi(\tau) = \gamma_{2}(\tau)$ and
          $\dot{\psi}(\tau) = \dot{\gamma_{2}}(\tau)$, for some $\tau > t_{c}$,
    \item $\psi \leq \gamma_{2}$.
  \end{enumerate}
\end{lemma}

\begin{proof}

\phantom{.}
\begin{outline}

  \begin{figure}
    \centering
    \includegraphics[scale=1]{figures/motion/rough/curvejoiningproof}
    \caption{Sketch of the proof of Lemma~\ref{lemma:curvejoining}.}%
    \label{fig:proof}
  \end{figure}

  % \1 Define $g$ such that $g(x, \tau) = \dot{\gamma}_{1}[\xi](\tau)$
  % for any $\xi$ such that $\gamma_{1}[\xi](\tau) = x$.

  % \2 Writing $h_{\tau}(\xi) := f_{x}(\xi, \tau)$, we define
  % $h_{\tau}^{-1}(x) := \{ \xi : f_{x}(\xi, \tau) = x \}$. Because this is the level set of a
  % continuous non-decreasing function, it must be a closed interval (Lemma~\ref{lemma:levelset}).
  % Hence, $g(x, \tau) := f_{v}(\max h_{\tau}^{-1}(x), \tau)$ is well-defined.

  \1 Identify for which parameters $\xi < t_{c} < \tau$ we have
  $\gamma_{1}[\xi](\tau) = \gamma_{2}(\tau)$ and
  $\dot{\gamma_{1}}[\xi](\tau) = \dot{\gamma}_{2}(\tau)$.

  \2 Define set $U$ and functions $X(t, x)$ and $\xi(t, x)$ as we did for
  $\gamma$ above, now for $\gamma_{1}$.

  \2 For each $\tau > t_{c}$, observe that $(\tau, \gamma_{2}(\tau)) \in U$.
  This means that there is some $\xi = \xi(\tau, \gamma_{2}(\tau))$ such that
  $\psi_{\tau} := \gamma_{1}[\xi]$ is the unique trajectory such that
  $\psi_{\tau}(\tau) = \gamma_{2}(\tau)$.
  %
  Therefore, we proceed to characterize the set $T$ of values of $\tau > t_{c}$
  for which also $\dot{\psi}_{\tau}(\tau) = \dot{\gamma}_{2}(\tau)$. More
  explicitly, we have
  $T := \{ \tau > t_{c} : \dot{\gamma}_{1}[\xi(\tau, \gamma_{2}(\tau))](\tau) = \dot{\gamma}_{2}(\tau) \}$.

  % Recall that Property~\ref{prop:xi-unique} ensures that for any
  % $\xi, \xi' \in h_{\tau}^{-1}(\gamma_{2}(\tau))$, we have
  % $\gamma_{1}[\xi] = \gamma_{1}[\xi']$.
  % %
  % Therefore, we will use $\xi(\tau) = \xi(\tau, \gamma_{2}(\tau))$ as a
  % canonical representative.

  \2 To this end, we define the auxiliary function
  $g(t, x) := \dot{\gamma}_{1}[\xi(t, x)](t)$, which gives the slope of the
  unique stopping trajectory through each point $(t, x) \in U$.

  \1 Function $g$ is continuous in $(t, x)$. We use the notation
  $N_{\varepsilon}(x) := (x - \varepsilon, x + \varepsilon)$.

  \2 We will write $f_{x}(\xi, t) = \gamma_{1}[\xi](t)$,
  $f_{v}(\xi, t) = \dot{\gamma}_{1}[\xi](t)$ and
  $h_{t}(\xi) = \gamma_{1}[\xi](t)$ to emphasize the quantities that we treat as
  variables. Observe that $h_{t}^{-1}(x) = X(t, x)$.

  \2 Let $x_{0} = f_{x}(\xi_{0}, \tau_{0})$ and $v_{0} = f_{v}(\xi_{0}, \tau_{0})$ for
  some $\xi_{0}$ and $\tau_{0}$ and pick some arbitrary $\varepsilon > 0$. Note that
  $\xi_{0} \in [\xi_{1}, \xi_{2}] := h_{\tau_{0}}^{-1}(x_{0})$.
  %
  We apply the $\varepsilon$-$\delta$ definition of continuity to each of the endpoints of this
  interval. Let $i \in \{1, 2\}$, then there exist $\delta_{i} > 0$ such that
  $\xi \in N_{\delta_{i}}(\xi_{i}), \tau \in N_{\delta_{i}}(\tau_{0})$ implies
  $f_{v}(\xi, \tau) \in N_{\varepsilon}(v_{0})$. Let $\delta = \min\{ \delta_{1}, \delta_{2} \}$ and define
  $N_{1} := (\xi_{1} - \delta, \xi_{2} + \delta)$ and $N_{2} := N_{\delta}(\tau_{0})$. Let $\xi \in N_{1}$
  and $\tau \in N_{2}$, then $f_{v}(\xi, \tau) \in N_{\varepsilon}(v_{0})$. This is obvious when $\xi$
  is chosen to be in one of $N_{\delta_{i}}(\xi_{i})$. Otherwise, we must have $\xi \in [\xi_{1}, \xi_{2}]$,
  in which case $f_{v}(\xi, \tau) = f_{v}(\xi_{1}, \tau) \in N_{\varepsilon}(v_{0})$.

  \2 Because $h_{\tau_{0}}(\xi)$ is continuous, the image $I := h_{\tau_{0}}(N_{1})$
  must be an interval containing $x_{0}$, with
  $\inf I = h_{\tau_{0}}(\xi_{1} - \delta)$ and
  $\sup I = h_{\tau_{0}}(\xi_{2} + \delta)$.

  \2 We argue that $I$ contains $x_{0}$ in its interior. For sake of
  contradiction, suppose $x_{0} = \max I$, then $h_{\tau_{0}}(\xi_{2} + \delta') = x_{0}$,
  for each $\delta' \in (0, \delta)$, because $h_{\tau_{0}}$ is non-decreasing, but this
  contradicts the definition of $\xi_{2}$. Similarly, when $x_{0} = \min I$, then
  $h_{\tau_{0}}(\xi_{1} - \delta') = x_{0}$, for each $\delta' \in (0, \delta)$, which contradicts the
  definition of $\xi_{1}$.

  \2 Define $\nu := \min \{ x_{0} - \inf I, \sup I - x_{0}\}$ and
  $N_{3} := (x_{0} - \nu / 2, x_{0} + \nu / 2)$. Because $h_{\tau}(\xi)$ is also
  continuous in $\tau$, there exists a neighborhood $N_{2}^{*} \subset N_{2}$ of
  $\tau_{0}$ such that for every $\tau \in N_{2}^{*}$, we have
  \begin{align*}
  &h_{\tau}(\xi_{1} - \delta) \leq h_{\tau_{0}}(\xi_{1} - \delta) + \nu/2 = \inf I + \nu /2 < x_{0} - \nu/2 , \\
  &h_{\tau}(\xi_{2} + \delta) \geq h_{\tau_{0}}(\xi_{2} + \delta) - \nu/2 = \sup I - \nu /2 > x_{0} + \nu/2 ,
  \end{align*}
  which shows that $h_{\tau}(N_{1}) \supset N_{3}$. It follows that
  $h_{\tau}^{-1}(N_{3}) \subset N_{1}$.

  \2 Finally, take any $\tau \in N_{2}^{*}$ and $x \in N_{3}$, then there exists
  some $\xi \in N_{1}$ such that $h_{\tau}(\xi) = x$ and
  $g(\tau, x) = f_{v}(\max h_{\tau}^{-1}(x), \tau) = f_{v}(\xi, \tau) \in N_{\varepsilon}(v_{0})$.

  \1 Function $g$ is non-decreasing and Lipschitz continuous in $x$.

  \2 Let $x_{1} \leq x_{2}$ and $\tau$ such that $g(\tau, x_{1})$ and
  $g(\tau, x_{2})$ are defined. There must be $\xi_{1} \leq \xi_{2}$ such that
  $h_{\tau}(\xi_{1}) = x_{1}$ and $h_{\tau}(\xi_{2}) = x_{2}$ and we have
  \begin{align*}
    g(\tau, x_{1}) = \dot{\gamma}_{1}[\xi_{1}](\tau)
    &= \max\{0, \dot{\gamma}_{1}(\xi_{1}) - \omega(\tau - \xi_{1}) \} \\
    &= \max\{0, \dot{\gamma}_{1}(\xi_{1}) - \omega(\xi_{2} - \xi_{1}) - \omega(\tau - \xi_{2}) \} \\
    &\leq \max\{0, \dot{\gamma}_{1}(\xi_{2}) - \omega(\tau - \xi_{2}) \}
    = \dot{\gamma}_{1}[\xi_{2}](\tau) = g(\tau, x_{2}) .
  \end{align*}

  \2 Furthermore, we have
  $\dot{\gamma}_{1}(\xi_{2}) \leq \dot{\gamma}_{1}(\xi_{1}) + \omega(\xi_{2} - \xi_{1})$,
  so that
  \begin{align*}
    g(\tau, x_{2}) &= \max\{0, \dot{\gamma}_{1}(\xi_{2}) - \omega(\tau - \xi_{2}) \} \\
    &\leq \max\{0, \dot{\gamma}_{1}(\xi_{1}) + \omega(\xi_{2} - \xi_{1}) - \omega(\tau - \xi_{2}) \} \\
    &= \max\{0, \dot{\gamma}_{1}(\xi_{1}) - \omega(\tau-\xi_{1}) + 2\omega(\xi_{2} - \xi_{1}) \} \\
    &\leq \max\{0, \dot{\gamma}_{1}(\xi_{1}) - \omega(\tau - \xi_{1}) \} + 2\omega(\xi_{2} - \xi_{1})
      = g(\tau, x_{1}) + 2\omega(\xi_{2} - \xi_{1}) .
  \end{align*}
  Together with the above non-decreasing property, this shows that $g$ is
  Lipschitz continuous in $x$, with Lipschitz constant $2\omega$.

  \1 Observe that $T$ can also be written as
  $T = \{ \tau > t_{c} : \dot{\gamma}_{2}(\tau) = g(\tau, \gamma_{2}(\tau)) \}$,
  so continuity of $g$ shows that it is a closed set
  (Lemma~\ref{lemma:levelset}). It is not necessarily connected (see Figure
  \dots), so it is the union of a sequence of disjoint closed intervals
  $T_{1}, T_{2}, \dots, T_{n}$.

  \2 Write $\tau_{i} := \min T_{i}$ and let
  $\psi_{i} := \gamma_{1}[\xi(\tau_{i}, \gamma_{2}(\tau_{i}))]$ be the unique stopping trajectory
  through $(\tau_{i}, \gamma_{2}(\tau_{i}))$.
  %
  By definition, we have $\dot{\gamma}_{2}(\tau) = g(\tau, \gamma_{2}(\tau)) = \dot{\psi}_{i}(\tau)$, for every $\tau \in T_{i}$.
  %
  Since $g(t, x)$ is continuous in $t$ and Lipschitz continuous in $x$, it is a
  consequence of the existence and uniqueness theorem (Picard-Lindel{\"o}f) that
  $\gamma_{2} = \psi_{i}$ on $T_{i}$.
  %
  Hence, we have $\gamma_{1}[\xi(\tau, \gamma_{2}(\tau))] = \psi_{i}$ for any
  $\tau \in T_{i}$, so $\psi_{i}$ can be seen as the canonical candidate
  trajectory corresponding to $T_{i}$.

  % Let $\tau'$ be the largest $\tau \in T_{i}$ such that
  % $\gamma_{2}(\tau) = \psi_{i}(\tau)$. Suppose $\tau' < \hat{\tau}_{i}$, then by
  % the existence and uniqueness theorem (Picard-Lindel{\"o}f) there exists a
  % unique solution of the initial value problem
  % \begin{align*}
  %   \dot{\phi}(t) = g(t, \phi(t)), \quad \phi(\tau') = \gamma_{2}(\tau')
  % \end{align*}
  % on the interval $[\tau' - \delta, \tau' + \delta]$ for some $\delta > 0$.
  %
  % But $\psi_{i}$ is a solution, so
  % $\psi_{i}(\tau' + \delta) = \gamma_{2}(\tau' + \delta)$ for some sufficiently
  % small $\delta > 0$, but this contradicts the definition of $\tau'$. Therefore,
  % $\tau' = \hat{\tau}_{i}$.

  \1 Show that $\psi_{1}$ exists. We write $s(\tau) := g(\tau, \gamma_{2}(\tau))$.

  \2 Case 1: Suppose $\gamma_{2}(t_{f}) \leq \gamma_{1}[t_{c}](t_{f})$, then because $g$ is
  non-decreasing in $x$, we have
  $g(t_{f}, \gamma_{2}(t_{f})) \leq g(t_{f}, \gamma_{1}[t_{c}](t_{f})) = \dot{\gamma}_{1}(t_{c}) - \omega(t_{f} - t_{c}) = 0$
  by definition of $t_{f}$, so $s(t_{f}) = 0$.

  \2 Case 2: Otherwise, (it follows from Lemma \dots) that $\gamma_{2}$ crosses
  $\gamma_{1}[t_{c}]$ at some time $t_{d} \in (t_{c}, t_{f})$, such that
  $\dot{\gamma}_{2}(t_{d}) > \gamma_{1}[t_{c}](t_{d}) = s(t_{d})$.

  \2 We have
  $\gamma_{1}[a_{1}](t) \leq \gamma_{2}(t) \leq \gamma_{1}[t_{c}](t)$
  for $t \in \{ t_{f}, t_{d} \}$, so the intermediate value theorem guarantees
  that $s(t)$ actually exists for the above two cases, because there is some
  $\xi$ such that
  $\gamma_{2}(t) = f_{x}(\xi, t) = \gamma_{1}[\xi](t)$ and thus
  $s(t) = g(t, \gamma_{2}(t)) = f_{v}(\max h_{t}^{-1}(\gamma_{2}(t)), t) = f_{v}(\xi, t) = \dot{\gamma}_{1}[\xi](t)$
  exists.

  \2 In both of the above two cases, we have
  $\dot{\gamma}_{2}(t_{c}) < \dot{\gamma}_{1}(t_{c}) = s(t_{c})$ and
  $\dot{\gamma}_{2}(t_{d}) \geq s(t_{d})$ for some $t_{d} \in \openhalf{t_{c}}{t_{f}}$.
  Hence, as a consequence of the intermediate value theorem, there must be some
  smallest $\tau_{1} \in \openhalf{t_{c}}{t_{d}}$ such that
  $\dot{\gamma_{2}}(\tau_{1}) = s(\tau_{1})$.

  \1 Show that $\psi_{i} \leq \gamma_{2}$ if and only if $i=1$.

  \2 Show that $\psi_{i}(t) > \gamma_{2}(t)$ for some $t$ implies $i \geq 2$.

  \3 Suppose $\psi_{i}$ crosses $\gamma_{2}$ at some $t_{x} \in (t_{c}, \tau_{i})$, so
  $\dot{\gamma}_{2}(t_{x}) < \dot{\gamma}_{1}[\xi_{i}](t_{x}) = s(t_{x})$.

  \3 By definition of $\tau_{i}$, we have $\dot{\gamma}_{2}(\tau_{i}) = s(\tau_{i})$.

  \3 Argue that we must have $\dot{\gamma}_{2}(t) > s(t)$ for some $t \in (t_{x}, \tau_{i})$, otherwise $\gamma_{2}(\tau_{i}) < \gamma_{1}[\xi_{i}](\tau_{i})$.

  \3 Therefore, this shows there exists $t \in (t_{x}, \tau_{i})$ such that
  $\dot{\gamma}_{2}(t) = s(t)$, which means that $i \geq 2$.

  \2 Show $i \geq 2$ implies $\psi_{i}(t) > \gamma_{2}(t)$ for some $t$.

  \2 In conclusion, $\psi := \psi_{1}$ is the unique trajectory satisfying the
  stated requirements.

  \qedhere

\end{outline}
\end{proof}


\begin{figure}
  \centering
  \includegraphics[scale=1]{figures/motion/rough/curvejoining}
  \caption{Two non-decreasing functions joined by a part with maximal negative
    curvature.}%
  \label{fig:curvejoining}
\end{figure}

\begin{remark}
  Note that condition $(i)$ in Lemma~\ref{lemma:curvejoining} is necessary, because Figure \dots shows
  a configuration in which the trajectories cannot be joined. Condition $(ii)$
  is not necessary, because Figure \dots shows a valid configuration that can be
  joined, but which does not satisfy this condition.
\end{remark}

Suppose we have two trajectories that cross each other exactly once. In the
following, we will investigate conditions under which, roughly speaking, these
trajectories can be glued together to form a smooth trajectory by introducing a
stopping trajectory in between, as illustrated in Figure~\ref{fig:curvejoining}.
%
Let $\gamma_{1} \in \mathcal{D}[a_{1}, b_{1}]$ and
$\gamma_{2} \in \mathcal{D}[a_{2}, b_{2}]$ and let
$[\xi, \tau] \subset [a_{1}, b_{1}] \cap [a_{2}, b_{2}]$ parameterize the interval over
which we introduce deceleration, then we define the candidate trajectory
\begin{align*}
  (\gamma_{1} * \gamma_{2})[\xi, \tau](t) :=
    \begin{cases}
      \gamma_{1}(t) &\text{ for } t \in [a_{1}, \xi] , \\
      \gamma_{1}[\xi](t) &\text{ for } t \in [\hspace{0.1em} \xi, \, \tau \hspace{0.05em} ] , \\
      \gamma_{2}(t) &\text{ for } t \in [\tau, b_{2}] .
    \end{cases}
\end{align*}
%
Suppose $\gamma_{1}$ and $\gamma_{2}$ intersect at exactly one time $t_{c}$, but do so
tangentially, i.e., such that $\dot{\gamma}_{1}(t_{c}) = \dot{\gamma}_{2}(t_{c})$, then
clearly $\gamma_{1} * \gamma_{2} := (\gamma_{1} * \gamma_{2})[t_{c}, t_{c}]$ satisfies
$\gamma_{1} * \gamma_{2} \in \mathcal{D}[a_{1}, b_{2}]$ and
$\gamma_{1} * \gamma_{2} \leq \min\{\gamma_{1}, \gamma_{2}\}$. Next, we present conditions under which we
can choose such $\xi$ and $\tau$ in the case when $\gamma_{1}$ and $\gamma_{2}$ do not
intersect tangentially.

When it exists, we will denote the unique $\psi$ from
Lemma~\ref{lemma:curvejoining} as $\gamma_{1} * \gamma_{2}$.


\begin{remark}
  Based on Lemma~\ref{lemma:curvejoining}, a numerical method could be developed
  to compute $\xi$ and $\tau$. However, the trajectories that we consider
  contain more structure that allows a simpler algorithm.
\end{remark}

\begin{lemma}
  Let $\gamma_{1} \in \mathcal{D}[a_{1}, b_{2}]$ and
  $\gamma_{2} \in \mathcal{D}[a_{2}, b_{2}]$ such that $\psi := \gamma_{1} * \gamma_{2}$ exists and
  let $\xi$ and $\tau$ denote the joining times. For all $\gamma \in \mathcal{D}[a, b]$ such
  that $\gamma \leq \min\{\gamma_{1}, \gamma_{2}\}$, we have $\gamma \leq \psi$.
\end{lemma}
\begin{proof}
  We obviously have $\gamma \leq \psi$ on $[a_{1}, \xi] \cup [\tau, b_{2}]$, so
  consider the interval $(\xi, \tau)$ of the joining deceleration part. Suppose
  there exists some $t_{d} \in (\xi, \tau)$ such that
  $\gamma(t_{d}) > \psi(t_{d})$. Because $\gamma(\xi) \leq \psi(\xi)$, this
  means that $\gamma$ must intersect $\psi$ at least once in
  $\halfopen{\xi}{t_{d}}$, so let
  $t_{c} := \sup \, \{ t \in \halfopen{\xi}{t_{d}} : \gamma(t) = \psi(t) \}$ be
  the latest time of intersection such that $\gamma \geq \psi$ on
  $[t_{c}, t_{d}]$. There must be some $t_{c} \in [t_{c}, t_{d}]$ such that
  $\dot{\gamma}(t_{v}) > \dot{\psi}(t_{v})$, otherwise
  \begin{align*}
    \gamma(t_{d}) = \gamma(t_{c}) + \int_{t_{c}}^{t_{d}} \dot{\gamma}(t) dt \leq \psi(t_{c}) + \int_{t_{c}}^{t_{d}} \dot{\psi}(t) dt = \psi(d_{t}) ,
  \end{align*}
  which contradicts our choice of $t_{d}$. Hence, for every
  $t \in [t_{v}, \tau]$, we have
  \begin{align*}
    \dot{\gamma}(t) \geq \dot{\gamma}(t_{v}) - \omega (t - t_{v}) > \dot{\psi}(t_{v}) - \omega(t - t_{v}) = \dot{\psi}(t) .
  \end{align*}
  It follows that $\gamma(\tau) > \psi(\tau)$, which contradicts
  $\gamma \leq \gamma_{2}$.
\end{proof}

\begin{figure}
  \centering
  \includegraphics[scale=1]{figures/motion/rough/bufferconstraint}
  \caption{Illustration of ``buffer constraint''.}%
  \label{fig:bufferconstraint}
\end{figure}


\begin{figure}
  \centering
  \includegraphics[scale=1]{figures/motion/rough/proof}
  \caption{Sketch of how the three boundaries are joined to form the
    optimal trajectory.}%
  \label{fig:proof}
\end{figure}


% \newpage
% \appendix
% \section{Miscellaneous}

\begin{lemma}\label{lemma:levelset}
  Let $f :\mathbb{R}^{n} \rightarrow \mathbb{R}^{m}$ be continuous and
  $y \in \mathbb{R}^{m}$, then the level set $N := f^{-1}(\{ y \})$ is a closed
  subset of $\mathbb{R}^{n}$.
\end{lemma}
\begin{proof}
  For any $y' \neq y$, there exists an open neighborhood $M(y')$ such that
  $y \notin M(y')$. The preimage $f^{-1}(M(y'))$ is open by continuity.
  Therefore, the complement
  $N^{c} = \{ x : f(x) \neq y \} = \cup_{y' \neq y} f^{-1}(\{y'\}) = \cup_{y' \neq y} f^{-1}(M(y'))$
  is open.
\end{proof}

\end{document}

% to enable the minted package
% Local Variables:
% TeX-command-extra-options: "-shell-escape"
% End:
