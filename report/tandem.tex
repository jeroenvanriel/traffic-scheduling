\documentclass[a4paper]{article}
\usepackage[margin=3.5cm, marginparsep=5pt, marginparwidth=80pt]{geometry}
\usepackage{amsmath}
\usepackage{amssymb}
\usepackage[svgnames]{xcolor}
\usepackage{amsthm}
% \makeatletter
% \def\th@plain{%
%   \thm@notefont{}% same as heading font
%   \itshape % body font
% }
% \def\th@definition{%
%   \thm@notefont{}% same as heading font
%   \normalfont % body font
% }
% \makeatother
\usepackage{dsfont}
\usepackage{graphicx}
\usepackage{caption}
\usepackage{hyperref}
\usepackage{cleveref}
\usepackage{datetime}
\usepackage{outlines}
\usepackage{float}
\usepackage{booktabs}
\usepackage{enumitem}
\usepackage{tabto}
\usepackage{mathtools}
\usepackage{nicematrix}
\usepackage{nccmath}

\usepackage{algorithm}
\usepackage{algpseudocode}

\usepackage{lipsum}
\usepackage{pdfcomment}
\usepackage[activate={true,nocompatibility},final,tracking=true,kerning=true,spacing=true,factor=500,stretch=15,shrink=15]{microtype}

\usepackage{apptools}
\AtAppendix{\counterwithin{lemma}{section}}

\addtolength{\skip\footins}{2mm}

\newcommand{\comment}[1]{\pdfmargincomment[author=Jeroen van Riel]{#1}}

% code highlighting
\usepackage{minted}
\usepackage{xpatch}
\newminted[cminted]{python}{fontsize=\small}
\xpretocmd{\cminted}{\RecustomVerbatimEnvironment{Verbatim}{BVerbatim}{}}{}{}

% link coloring
\hypersetup{
   colorlinks,
   linkcolor={red!90!black},
   citecolor={green!40!black},
   urlcolor={blue!60!black}
}

% concatenation symbol (c.f. ++ in Haskell)
\newcommand\mdoubleplus{\mathbin{+\mkern-10mu+}}

% end of proof symbol
\newcommand{\newmarkedtheorem}[1]{%
  \newenvironment{#1}
    {\pushQED{\qed}\csname inner@#1\endcsname}
    {\popQED\csname endinner@#1\endcsname}%
  \newtheorem{inner@#1}%
}
% \renewenvironment{proof}{{\noindent\bfseries Proof.}}{*something*}
%\let\oldproofname=\proofname
%\renewcommand{\proofname}{\rm\bf{\oldproofname}}


\theoremstyle{definition}
%\newtheorem{eg}{Example}[section]
\newmarkedtheorem{eg}{Example}[section]
\newtheorem{remark}{Remark}
\theoremstyle{plain}
\newtheorem{define}{Definition\hspace{0.25em}\ignorespaces}
\newtheorem{property}{Property\hspace{0.25em}\ignorespaces}
\newtheorem{observation}{Observation}
\newtheorem{proposition}{Proposition}
\newtheorem{lemma}{Lemma\hspace{0.25em}\ignorespaces}
\newtheorem{corollary}{Corollary}
\newtheorem{theorem}{Theorem\hspace{0.25em}\ignorespaces}
\newtheorem{assump}{Assumption\hspace{0.25em}\ignorespaces}

\DeclareMathOperator{\interior}{int}
\DeclareMathOperator*{\argmax}{arg\,max}
\DeclareMathOperator*{\argmin}{arg\,min}
\newcommand*\diff{\mathop{}\!\mathrm{d}}

% cref settings
\crefname{equation}{equation}{equations}
\newcommand{\crefrangeconjunction}{--}

\newdateformat{monthyeardate}{\monthname[\THEMONTH] \THEYEAR}

\author{Jeroen van Riel}
\date{\monthyeardate\today}

\title{Vehicle trajectory planning in a single lane with minimum following
  distance and boundary conditions}

\begin{document}

\maketitle

\begin{abstract}
  This section considers a model of a single-lane road on which overtaking is
  not allowed.
  %
  Vehicles are modeled as double integrators with bounds on speed and
  acceleration. Consecutive vehicles must keep some fixed \textit{following
    distance} to avoid collisions.
  %
  It is assumed that vehicles enter and exit the lane at
  predetermined \textit{schedule times}. Whenever a vehicle enters or exits, it
  must drive at full speed.
  %
  For an optimization objective that, roughly speaking, minimizes the distance
  to the end of the lane at all times, we present an algorithm to compute
  optimal trajectories.
  %
  Assuming a fixed minimum lane length, we characterize feasibility of this
  trajectory planning problem in terms of a system of linear inequalities
  involving only the schedule times.
  %
\end{abstract}

% \tableofcontents

\newcommand\halfopen[2]{\ensuremath{[#1,#2)}}
\newcommand\openhalf[2]{\ensuremath{(#1,#2]}}

\renewcommand{\labelitemii}{\textbullet}
\renewcommand{\labelitemiii}{\textbullet}

\section{Trajectory planning problem}

Vehicles are modeled as double integrators with bounded speed and acceleration,
which means that we only consider their longitudinal position on the road. Let
$\mathcal{D}[a,b]$ denote the set of valid \emph{trajectories}, which we define to be
all continuously differentiable functions $x : [a,b] \rightarrow \mathbb{R}$ satisfying
the constraints
\begin{align}
  0 \leq \dot{x}(t) \leq 1 \quad \text{ and } \quad
  {-\omega} \leq \ddot{x}(t) \leq \bar{\omega} , \quad \text{ for all } t \in [a,b] ,
\end{align}
for some fixed acceleration bounds $\omega, \bar{\omega} > 0$. Note that we use $\dot{x}$
and $\ddot{x}$ to denote the first and second derivative with respect to time
$t$. When we have a general positive speed upper bound, we can always apply an
appropriate scaling of the time axis and the acceleration bounds to obtain the
form.
%
Consider positions $A, B \in \mathbb{R}$, such that $B \geq A$, which denote the
start\footnote{We could have assumed $A=0$, but we will later piece together
  multiple lanes to model intersections.} and end position of the lane.
%
Let $\bar{D}[a, b] \subset \mathcal{D}[a, b]$ denote the set of trajectories $x$ that
satisfy the boundary conditions
\begin{align}
  x(a) = A \; \text{ and } \; x(b) = B .
\end{align}
Even further, let $D[a,b] \subset \bar{D}[a, b]$ induce the boundary conditions
\begin{align}
  \dot{x}(a) = \dot{x}(b) = 1 .
\end{align}
In words, these boundary conditions require that a vehicle arrives to and
departs from the lane at predetermined times $a$ and $b$ and do so at full
speed.

Let $L > 0$ denote the required \textit{following distance} between consecutive
vehicles.
%
Suppose we have $N$ vehicles that are scheduled to traverse the lane. For each
vehicle $i$, let $a_{i}$ and $b_{i}$ denote the \textit{schedule time} for entry
and exit, respectively. A feasible solution consists of a sequence of
trajectories $x_{1}, \dots, x_{N}$ such that
\begin{subequations}\label{eq:feasibility}
\begin{align}
x_{i} \in D[a_{i}, b_{i}] \quad \quad & \text{ for each } i, \\
x_{i} \leq x_{i-1} - L \quad \quad &\text{ for each } i \geq 2,
\end{align}
\end{subequations}
%
where we use the shorthand notation $\gamma_{1} \leq \gamma_{2}$ to mean
$\gamma_{1}(t) \leq \gamma_{2}(t)$ for all
$t \in [a_{1}, b_{1}] \cap [a_{2}, b_{2}]$, given some
$\gamma_{1} \in \mathcal{D}[a_{1}, b_{1}]$ and
$\gamma_{2} \in \mathcal{D}[a_{2}, b_{2}]$.
% We use $\gamma \leq \min \{ \gamma_{1}, \dots, \gamma_{n} \}$ as a shorthand for $\gamma \leq \gamma_{1}, \dots, \gamma \leq \gamma_{n}$.

As optimization objective, we will consider
\begin{align}\label{eq:objective}
  \max \; \sum_{i=1}^{N} \int_{a_{i}}^{b_{i}} x_{i}(t) \diff t ,
\end{align}
which, roughly speaking, seeks to keep all vehicles as close to the end of the
lane at all times.
%
In particular, we will show that optimal trajectories can be understood as the
concatenation of at most four different types of trajectory parts. Based on this
observation, we present an algorithm to compute optimal trajectories.
%
Assuming the system parameters $(\omega, \bar{\omega},A,B,L)$ to be fixed, with lane
length $B-A$ sufficiently large, we will show that feasibility of the trajectory
optimization problem is completely characterized by a system of linear
inequalities in terms of all schedule times $a_{i}$ and $b_{i}$.

Note that objective~\eqref{eq:objective} does not capture energy efficiency in
any way. Although this is not desirable in practice, it is precisely this
assumption that enables the analysis in this section. Deriving a similar
characterization of feasibility and optimal trajectories under an objective that
does model energy concumption is an interesting topic for further research.

% \subsection{Direct transcription}

% A straightforward way of solving
% problem~\eqref{eq:feasibility}--\eqref{eq:objective} is through direct
% transcription to a mixed-integer linear program by only considering the vehicle
% trajectories $x_{i}(t)$ at discrete time steps $t_{1}, t_{2}, \dots, t_{n}$.
% %
% We can use the forward Euler scheme to discretize the constraints.
% %
% The downside of this technique is that its computational complexity scales very
% poorly with the number of time steps.

\section{Single vehicle problem}

We will first consider a somewhat generalized version of the constraints~\eqref{eq:feasibility} for
a single vehicle $i$. Therefore, we lighten the notation slightly by dropping
the vehicle index $i$ and instead of $x_{i-1} - L$, we assume we are given some
arbitrary \emph{lead vehicle boundary} $\bar{x} \in \bar{D}[\bar{a}, \bar{b}]$, then we
consider the optimization problem
\begin{align}
  \max_{x \in D[a, b]} \int_{a}^{b} x(t) \diff t \quad \text{ such that } \quad x \leq \bar{x} .
\end{align}
to which we will refer as the \emph{single vehicle problem}.

\subsection{Necessary conditions}

For every trajectory $x \in D[a,b]$, we derive two upper
bounding trajectories $x^{1}$ and $\hat{x}$ and one lower bounding trajectory
$\check{x}$, see Figure~\ref{fig:necessary-conditions}.
%
Using these bounding trajectories, we will then formulate four necessary
conditions for feasibility of the single vehicle problem.

Let the \emph{full speed boundary} $x^{1}$ be defined as
\begin{align}
  x^{1}(t) = A + t - a,
\end{align}
for all $t \in [a, b]$, then we clearly have $x \leq x^{1}$.
%
Next, since deceleration is at most $\omega$, we have
$\dot{x}(t) \geq \dot{x}(a) - \omega(t - a) = 1 - \omega(t - a)$, which we
combine with the speed constraint $\dot{x} \geq 0$ to derive
$\dot{x}(t) \geq \max\{0, 1 - \omega (t - a) \}$. Hence, we obtain the lower
bound
\begin{align}\label{eq:check-x}
  x(t) = x(a) + \int_{a}^{t} \dot{x}(\tau) \diff \tau \geq A + \int_{a}^{t} \max\{0, 1 - \omega (\tau - a) \} \diff \tau =: \check{x}(t) .
\end{align}
%
Analogously, we derive an upper bound from the fact that acceleration is at most $\bar{\omega}$. Observe that we have
$\dot{x}(t) + \bar{\omega} (b - t) \geq \dot{x}(b) = 1$, which we combine
with the speed constraint $\dot{x}(t) \geq 0$ to derive
$\dot{x}(t) \geq \max \{ 0, 1 - \bar{\omega}(b - t) \}$. Hence, we obtain the
upper bound
\begin{align}\label{eq:hat-x}
  x(t) = x(b) - \int_{t}^{b} \dot{x}(\tau) \diff \tau
  \leq B - \int_{t}^{b} \max\{ 0, 1 -\bar{\omega} (b - \tau) \} \diff \tau =: \hat{x}(t) .
\end{align}
%
We refer to $\check{x}$ and $\hat{x}$ as the \emph{entry boundary} and
\emph{exit boundary}, respectively.


\begin{figure}
  \centering
  \includegraphics[scale=1]{figures/motion/rough/necessary-conditions}
  \caption{Illustration of the four bounding trajectories
    $\bar{x}, x^{1}, \hat{x}, \check{x}$ that bound feasible trajectories from
    above and below. We also drew an example of a feasible trajectory $x$ in
    blue. The horizontal axis represents time and the vertical axis corresponds
    to the position on the lane, so the vertical dashed grey lines correspond to
    the start and end of the lane.}%
  \label{fig:necessary-conditions}
\end{figure}


\begin{lemma}\label{lemma:necessary-conditions}
  Let $\bar{x} \in \bar{D}[\bar{a}, \bar{b}]$ and assume there exists a
  trajectory $x \in D[a, b]$ such that $x \leq \bar{x}$, then the following
  conditions must hold \TabPositions{3cm}
  \begin{enumerate}[label=(\roman*)\quad,leftmargin=5em]
    \item $b-a \geq B-A$, \tab (full speed constraint)
    \item $\bar{b} \leq b$, \tab (downstream order constraint)
    \item $\bar{a} \leq a$, \tab (upstream order constraint)
    \item $\bar{x} \geq \check{x}$. \tab (entry space constraint)
  \end{enumerate}
\end{lemma}
\begin{proof}
  Each of the conditions corresponds somehow to one of the four bounding
  trajectories defined above.
  %
  Observe that $x^{1}(t) = B$ for $t = a + (B-A)$, which can be interpreted as
  the earliest time of departure from the lane. This shows that
  $b \geq a + (B-A)$, which is equivalent with (i).
  %
  % When (i) is violated, $x$ must cross $x^{1}$ somewhere, which means that the
  % maximum speed constraint must be violated.
  When either (ii) or (iii) is violated, the constraint $x \leq \bar{x}$ conflicts
  with one of the boundary conditions $x(a) = A$ or $x(b) = B$. To see that (iv)
  must hold, suppose that $\bar{x}(\tau) < \check{x}(\tau)$ for some time $\tau$. Since
  $\bar{a} \leq a$, this means that $\bar{a}$ must intersect $\check{a}$ from
  above. Therefore, any trajectory that satisfies $x \leq \bar{x}$ must also
  intersect $\check{a}$ from above, which contradicts the assumption that $x$
  was a feasible solution.
\end{proof}

We show that the boundaries $\hat{x}$ and $\check{x}$ together could yield yet
another necessary condition. It is straightforward to verify
from~\cref{eq:hat-x,eq:check-x} that $\hat{x}(t) \geq B - 1/(2\bar{\omega})$ and
$\check{x}(t) \leq A + 1/(2\omega)$. Therefore, whenever
$B - A < 1/(2\bar{\omega}) + 1/(2\omega)$, these boundaries intersect for certain values
of $a$ and $b$. Because the exact condition is somewhat cumbersome to
characterize, we avoid this case by simply assuming that the lane length is
sufficiently large.

\begin{assump}\label{eq:AB-assumption}
  The length of the lane satisfies $B - A \geq 1/(2\omega) + 1/(2\bar{\omega})$.
\end{assump}

\pagebreak

\subsection{Constructing the optimal trajectory}

Assuming the four conditions of Lemma~\ref{lemma:necessary-conditions} hold, we will construct an optimal
solution $x^{*}$ for the single vehicle problem, thereby showing that these
conditions are thus also sufficient for feasibility.
%
First, we construct $x^{*}$ by combining the upper boundaries $\bar{x}$,
$\hat{x}$ and $x^{1}$ in a certain way to obtain a smooth trajectory satisfying
$x^{*} \in D[a,b]$.
%
We show that $x^{*}$ is still an upper boundary for any other feasible solution,
which shows that it is optimal.

\begin{figure}
  \centering
  \includegraphics[scale=1]{figures/motion/rough/proof}
  \caption{The minimum boundary $\gamma$, induced by three upper boundaries
    $\bar{x}$, $\hat{x}$ and $x^{1}$, is smoothened around time $t_{1}$ and
    $t_{2}$, where the derivative is discontinuous, to obtain the smooth optimal
    trajectory $x^{*}$, drawn in green. The times $\xi_{i}$ and $\tau_{i}$
    correspond to the start and end of the connecting deceleration as
    defined in Section~\ref{sec:smoothing}.}%
  \label{fig:optimal-construction}
\end{figure}

The starting point of the construction is the \emph{minimum boundary}
$\gamma : [a,b] \rightarrow \mathbb{R}$ induced by the upper boundaries, defined as
\begin{align}\label{eq:min-boundary}
  \gamma(t) := \min \{ \bar{x}(t), \hat{x}(t), x^{1}(t) \} .
\end{align}
%
Obviously, $\gamma$ is still a valid upper boundary for any feasible solution,
%
but in general, $\gamma$ may have a discontinuous derivative at some\footnote{In fact, it can be shown that, under the necessary conditions, there are at most two of such discontinuities.} isolated
points in time.

\begin{define}\label{def:piecewise-trajectory}
  Let $\mathcal{P}[a,b]$ be the set of functions $\mu : [a, b] \rightarrow \mathbb{R}$ for
  which there is a finite subdivision $a = t_{0} < \cdots < t_{n+1} = b$ such that
  the truncation $\mu|_{[t_{i}, t_{i+1}]} \in \mathcal{D}[t_{i}, t_{i+1}]$ is
  a smooth trajectory, for each $i \in \{0, \dots, n\}$, and for which the one-sided limits of $\dot{\mu}$ satisfy
  \begin{align}
    \dot{\mu}(t_{i}^{-}) := \lim_{t \uparrow t_{i}} \dot{\mu}(t) > \lim_{t \downarrow t_{i}} \dot{\mu}(t) =: \dot{\mu}(t_{i}^{+}) ,
  \end{align}
  for each $i \in \{1, \dots, n\}$. We refer to such $\mu$ as a \emph{piecewise
    trajectory (with downward bends)}.
\end{define}

It is not difficult to see from Figure~\ref{fig:necessary-conditions} that,
under the necessary conditions, $\gamma$ satisfies the above definition, so
$\gamma \in \mathcal{P}[a,b]$. In other words, $\gamma$ consists of a number of
pieces that are smooth and satisfy the vehicle dynamics, with possibly some
sharp bend downwards where these pieces come together.
%
Next, we present a simple procedure to smoothen out this kind of discontinuity
by decelerating from the original trajectory somewhat before some $t_{i}$, as
illustrated in Figure~\ref{fig:optimal-construction}. We show that this procedure can be repeated at every
point of discontinuity.

\subsubsection{Deceleration boundary}
In order to formalize the smoothing procedure, we will first define some
parameterized family of functions to model the deceleration part.
%
Recall the derivation of $\check{x}$ in equation~\eqref{eq:check-x} and the discussion
preceding it, which we will now generalize a bit.
%
Let $x \in \mathcal{D}[a, b]$ be some smooth trajectory, then observe that $\dot{x}(t) \geq \dot{x}(\xi) - \omega(t - \xi)$ for all $t \in [a, b]$.
Combining this with the constraint $\dot{x}(t) \in [0, 1]$, this yields
\begin{align}
  \dot{x}(t) \geq \max\{ 0, \, \min\{1, \, \dot{x}(\xi) - \omega (t - \xi) \}\} =: \{\dot{x}(\xi) - \omega(t-\xi)\}_{[0,1]} ,
\end{align}
where use $\{ \cdot \}_{[0,1]}$ as a shorthand for this clipping operation.
%
Hence, for any $t \in [a,b]$, we obtain the following lower bound
\begin{align}\label{eq:deceleration-boundary}
  x(t) = x(\xi) + \int_{\xi}^{t} \dot{x}(\tau) \diff \tau \geq x(\xi) + \int_{\xi}^{t} \{\dot{x}(\xi) - \omega(\tau - \xi)\}_{[0,1]} \diff \tau =: x[\xi] (t) ,
\end{align}
where we will refer to the right-hand side as the \emph{deceleration boundary} of $x$
at $\xi$.
%
Observe that this definition indeed generalizes the definition of $\check{x}$,
because we have $\check{x}=(x[a])|_{[a,b]}$, so $x[a]$ restricted to the
interval $[a,b]$.

Note that $x[\xi]$ depends on $x$ only through the two real numbers $x(\xi)$ and
$\dot{x}(\xi)$. It will be convenient later to rewrite the right-hand side
of~\eqref{eq:deceleration-boundary} as
\begin{align}
  x^{-}[p, v, \xi](t) = p + \int_{\xi}^{t} \{ v - \omega(\tau - \xi) \}_{[0,1]} \diff \tau ,
\end{align}
such that $x[\xi](t) = x^{-}[x(\xi), \dot{x}(\xi), \xi](t)$.
%
We can expand the integral in this expression further by carefully handling the
clipping operation. Observe that the expression within the clipping operation
reaches the bounds $1$ and $0$ for $\delta_{1} := \xi - (1-v)/\omega$ and
$\delta_{0} := \xi + v/\omega$, respectively. Using this notation, a straightforward
calculation shows that
\begin{align}\label{eq:x-}
  x^{-}[p,v,\xi](t) = p +
  \begin{cases}
    {(1 - v)}^{2} / (2\omega) + (t - \xi) & \text{ for } t \leq \delta_{1} , \\
    v(t - \xi) - \omega{(t-\xi)}^{2} /2 & \text{ for } t \in [\delta_{1} , \delta_{0}] , \\
    v^{2}/(2\omega) & \text{ for } t \geq \delta_{0} .
  \end{cases}
\end{align}
%
It is easily verified that these three cases coincide at
$t \in \{\delta_{1}, \delta_{0}\}$, which justifies the overlap there.
Furthermore, since $x$ and $\dot{x}$ are continuous by assumption, this shows
that $x[\xi](t) = x^{-}[x(\xi), \dot{x}(\xi), \xi](t)$ is continuous as a
function of either of its arguments\footnote{Even more, it can be shown that
  $x[\xi](t)$ is continuous as a function of $(\xi, t)$.}.
%
Assuming $0 \leq v \leq 1$, it can be verified that for every $t \in \mathbb{R}$, we
have $\ddot{x}^{-}[p,v,\xi](t) \in \{-\omega, 0\}$ and
$\dot{x}^{-}[p,v,\xi](t) \in [0,1]$ due to the clipping operation, so that
$x^{-}[p,v,\xi] \in \mathcal{D}(-\infty,\infty)$.

\begin{figure}
  \centering
  \includegraphics[scale=1.1]{figures/motion/rough/deceleration-boundary}
  \caption{Illustration of some piecewise trajectory $\mu \in \mathcal{P}[a,b]$ with
    some deceleration boundary $\mu[\xi]$ at time $\xi$ in blue and the unique
    connecting deceleration $\mu[\xi_{1}]$ in green. We truncated both
    deceleration boundaries for a more compact figure. This particular $\mu$ has a
    discontinuous derivative at times $t_{1}$ and $t_{2}$. The careful observer
    may notice that $\mu$ cannot occur as the minimum boundary defined in~\eqref{eq:min-boundary}, but
    please note that the class of piecewise trajectories $\mathcal{P}[a,b]$ is
    just slightly more general than necessary for our current purposes.}%
  \label{fig:deceleration-boundary}
\end{figure}


\paragraph{Piecewise trajectories.}
Let $\mu \in \mathcal{P}[a, b]$ be some piecewise trajectory and let
$a = t_{0} < \cdots < t_{n+1} = b$ denote the corresponding subdivision as in
Definition~\ref{def:piecewise-trajectory}, then we generalize the definition of
a deceleration boundary to $\mu$.
%
Whenever $\xi \in [a,b] \setminus \{ t_{1}, \dots, t_{n}\}$, we just define
$\mu[\xi] := x^{-}[\mu(\xi), \dot{\mu}(\xi), \xi]$.
%
However, when $\xi \in \{ t_{1}, \dots, t_{n}\}$, the derivative
$\dot{\mu}(\xi)$ is not defined, so we to use the left-sided limit instead, by
defining $\mu[\xi] := x^{-}[\mu(\xi), \dot{\mu}(\xi^{-}), \xi]$.

\begin{remark}\label{rem:lower-bound-piecewise}
  Please note that we cannot just replace $x$ with $\mu$ in inequality~\eqref{eq:deceleration-boundary} to
  obtain a similar bound for $\mu$ on the its full interval $[a,b]$.
  Instead, we get the following \emph{piecewise lower bounding} property.
  %
  Consider some interval
  $I \in \{ [a, t_{1}], \openhalf{t_{1}}{t_{2}}, \dots, \openhalf{t_{n}}{b} \}$, then what
  remains true is that $\xi \in I$ implies $\mu(t) \geq \mu[\xi](t)$ for every $t \in I$.
\end{remark}


\subsubsection{Smoothing procedure}\label{sec:smoothing}
Let $\mu \in \mathcal{P}[a,b]$ be some piecewise trajectory and let
$a = t_{0} < \cdots < t_{n+1} = b$ denote the subdivision as in Definition~\ref{def:piecewise-trajectory}.
%
We first show how to smoothen the discontinuity at $t_{1}$ and then argue how to
repeat this process for the remaining times $t_{i}$.

\begin{assump}\label{assump2}
  Throughout the following discussion, we assume $\mu \geq \mu[a]$ and $\mu \geq \mu[b]$.
\end{assump}

Our aim is to choose some time $\xi \in [a,t_{1}]$ from which the vehicle starts
fully decelerating, such that $\mu[\xi] \leq \mu$ and such that $\mu[\xi]$ touches $\mu$ at some time
$\tau \in [t_{1}, b]$ tangentially.
%
We will show there is a unique trajectory $\mu[\xi]$ that satisfies these requirements
and refer to it as the \emph{connecting deceleration}, see
Figure~\ref{fig:deceleration-boundary} for an example.


\paragraph{Touching.}
Recall Remark~\ref{rem:lower-bound-piecewise}, which asserts that we have
$\mu[\xi] \leq \mu$ on $[a,t_{1}]$ for any $\xi \in [a, t_{1}]$.
%
After the discontinuity, so on the interval $[t_{1}, b]$, we want
$\mu[\xi] \leq \mu$ and equality at least somewhere, so we measure the relative
position of $\mu[\xi]$ with respect to $\mu$ here, by considering
\begin{align}
  d(\xi) := \min_{t \in [t_{1}, b]} \mu(t) - \mu[\xi](t) .
\end{align}
Since $\mu(t)$ and $\mu[\xi](t)$ are both continuous as a function of $t$ on the interval
$[t_{1}, b]$, this minimum actually exists (extreme value theorem).
%
Furthermore, since $d$ is the minimum of a continuous function over a closed
interval, it is continuous as well (see Lemma~\ref{lemma:inf-continuous}).
%
Observe that $d(a) \geq 0$, because $\mu \geq \mu[a]$ by assumption.
%
By definition of $t_{1}$, we have $\dot{\mu}(t_{1}^{-}) > \dot{\mu}(t_{1}^{+})$,
from which it follows that $\mu(t) < \mu[t_{1}](t)$ for $ t\in (t_{1}, t_{1} + \epsilon)$ for some
small $\epsilon > 0$, which shows that $d(t_{1}) < 0$.
%
By the intermediate value theorem, there is $\xi_{1} \in \halfopen{a}{t_{1}}$ such
that $d(\xi_{1}) = 0$.


\paragraph{Uniqueness.}
It turns out that $\xi_{1}$ itself is not necessarily unique, which we explain
below. Instead, we are going to show that the connecting deceleration
$\mu[\xi_{1}]$ is unique. More precisely, given any other $\xi \in \halfopen{a}{t_{1}}$ such
that $d(\xi) = 0$, we will show that $\mu[\xi] = \mu[\xi_{1}]$.

The first step is to establish that the level set
\begin{align}
  X := \{ \xi \in \halfopen{a}{t_{1}} : d(\xi) = 0 \}
\end{align}
is a closed interval. To this end, we show that $d$ is non-increasing on
$\halfopen{a}{t_{1}}$, which together with continuity implies the desired result
(see Lemma~\ref{lemma:levelset}).
%
To show that $d$ is non-increasing, it suffices to show that $\mu[\xi](t)$ is
non-decreasing as a function of $\xi$, for every $t \in [t_{1}, b]$.
%
We can do this by computing the partial derivative of $\mu[\xi]$ with respect to $\xi$
and verifying it is non-negativity.
%
Recall the definition of $\mu[\xi]$, based on $x^{-}$ in equation~\eqref{eq:x-}.
%
Using similar notation, we write
$\delta_{1}(\xi) = \xi - (1 - \dot{\mu}(\xi))/\omega$ and
$\delta_{0}(\xi) = \xi + \dot{\mu}(\xi)/\omega$ and compute
%
\begin{align}
  \frac{\partial}{\partial \xi} \mu[\xi](t) =
  \dot{\mu}(\xi) +
  \begin{cases}
    \ddot{\mu}(\xi)(\dot{\mu}(\xi)-1)/\omega - 1 &\text{ for } t \leq \delta_{1}(\xi) , \\
    \ddot{\mu}(\xi)(t-\xi) - \dot{\mu}(\xi) + \omega(t-\xi) &\text{ for } t \in [\delta_{1}(\xi),\delta_{0}(\xi)] , \\
    \ddot{\mu}(\xi)\dot{\mu}(\xi)/ \omega &\text{ for } t \geq \delta_{0}(\xi) .
  \end{cases}
\end{align}
%
It is easily verified that the cases match at
$t \in \{\delta_{1}(\xi), \delta_{0}(\xi)\}$, which justifies the overlaps there.
%
Consider any $\xi \in \halfopen{a}{t_{1}}$ and $t \in [t_{1}, b]$, then we always have
$\delta_{1}(\xi) \leq \xi \leq t$, so we only have to verify the second and third case:
%
\begin{subequations}
\begin{alignat}{2}
  \frac{\partial}{\partial \xi} \mu[\xi](t) &= (\ddot{\mu}(\xi) + \omega)(t-\xi) \geq 0 \quad\quad\quad &&\text{ for } t \in [\delta_{1}(\xi) ,\delta_{0}(\xi)], \label{eq:case2} \\
  \frac{\partial}{\partial \xi} \mu[\xi](t) &\geq \dot{\mu}(\xi) + (-\omega)\dot{\mu}(\xi)/\omega = 0 &&\text{ for } t \geq \delta_{0}(\xi) . \label{eq:case3}
\end{alignat}
\end{subequations}
%
This concludes the argument for $X$ being a closed interval.

Assuming $\xi$ to be fixed, observe that there is equality in~\eqref{eq:case2} for some
$t \in [\delta_{1}(\xi), \delta_{0}(\xi)]$ if and only if there is equality in~\eqref{eq:case3}
for some other $t' \geq \delta_{0}(\xi)$. Note that this happens precisely when
$\ddot{\mu}(\xi) = -\omega$.
%
Therefore, whenever $\mu$ is fully deceleration, so $\dot{\mu}(t)=-\omega$ on
some open interval $U \subset (a, t_{1})$, we have
$(\partial/\partial \xi) \mu[\xi](t) = 0$ for all $t \geq \delta_{1}(\xi)$.
%
This essentially means that any choice of $\xi \in U$ produces the same
trajectory $\mu[\xi]$. Please see Figure~\ref{fig:deceleration-boundary-unique}
for an example of this case.
%
This observation is key to the remaining uniqueness argument.

\begin{figure}[h]
  \centering
  \vspace*{0.5em}
  \includegraphics[scale=1.1]{figures/motion/rough/deceleration-boundary-unique}
  \caption{Example of a piecewise trajectory $\mu$ with a part of full
    deceleration over some interval $U$ such that any choice of $\xi \in U$
    produces the same deceleration boundary $\mu[\xi]$, which naturally
    coincides with $\mu$ on $U$.}
  \label{fig:deceleration-boundary-unique}
\end{figure}

Since $X$ is a closed interval, we may define $\xi_{0} = \min X$.
%
Consider any $\xi' \in X$ with $\xi' > \xi_{0}$, then we show $\mu[\xi'](t) = \mu[\xi_{0}](t)$ for all
$t \in [\xi_{0}, b]$. For sake of contradiction, suppose there is some $t' \in [\xi_{0}, b]$ such that
$\mu[\xi'](t') > \mu[\xi_{0}](t')$, then there must be some open interval $U \subset (\xi_{0}, \xi')$ such that
\begin{align}\label{eq:positive-partial-derivative}
  \frac{\partial}{\partial \xi}\mu[\xi](t') > 0 \text{ for all } \xi \in U .
\end{align}
However, we argued in the previous paragraph that this actually holds for any
$t' \geq \delta_{1}(\xi)$.
%
In particular, let $t^{*} \in [t_{1}, b]$ be such that
$\mu(t^{*}) = \mu[\xi_{0}](t^{*})$, then
$t^{*} \geq t_{1} \geq \xi \geq \delta_{1}(\xi)$,
so~\eqref{eq:positive-partial-derivative} yields $\mu[\xi'](t^{*}) > \mu[\xi_{0}](t^{*})$, but then
$d(\xi') > d(\xi_{0}) = 0$, so $\xi' \notin X$, a contradiction.


\paragraph{Touching tangentially.}
\comment{I might add a figure for this whole part. Especially for the case tau1
  = b, because feasibility of the resulting trajectory relies on this.}
%
It remains to show that $\mu$ and $\mu[\xi_{0}]$ touch tangentially somewhere on
$[t_{1}, b]$. Let $\tau_{1} \in [t_{1}, b]$ be the smallest time such that
$\mu(\tau_{1}) - \mu[\xi_{0}](\tau_{1}) = d(\xi_{0}) = 0$ and consider the
following three cases.

First of all, note that $\tau_{1} = t_{1}$ is not possible, because this
would require
\begin{align}
  \dot{\mu}(t_{1}^{+}) > \dot{\mu}[\xi_{0}](t_{1}^{+}) = \dot{\mu}[\xi_{0}](t_{1}) ,
\end{align}
but since $\mu$ is a piecewise trajectory, we must have
$\dot{\mu}(t_{1}^{-}) > \dot{\mu}(t_{1}^{+}) > \dot{\mu}[\xi_{0}](t_{1})$. This
shows that $\mu(t_{1} - \epsilon) < \mu[\xi_{0}](t_{1} - \epsilon)$, for
some small $\epsilon > 0$, which contradicts $\mu[\xi_{0}] \leq \mu$.

Suppose $\tau_{1} \in (t_{1}, b)$, then recall the definition of $d(\xi_{0})$
and observe that the usual first-order necessary condition (derivative zero) for
local minima requires $\dot{\mu}(\tau_{1}) = \dot{\mu}[\xi_{0}](\tau_{1})$.
\comment{I find this argument clear enough, but I am not sure if
  ''first-order necessary condition'', needs some further elaboration. It is
  just when you are minimizing some differentiable function f(x), then a local
  minimum a should have f'(a) = 0.}

Finally, consider $\tau_{1} = b$.
%
Observe that $\dot{\mu}(b) > \dot{\mu}[\xi_{0}](b)$, would contradict
minimality of $\tau_{1} = b$. Therefore, suppose
$\dot{\mu}(b) < \dot{\mu}[\xi_{0}](b)$, then
$\dot{\mu}[b](b) = \dot{\mu}(b) < \dot{\mu}[\xi_{0}](b)$, so
\begin{align}
  \dot{\mu}[b](t) \leq \dot{\mu}[\xi_{0}](t) \text{ for } t \leq b ,
\end{align}
but then $\mu[b](t) > \mu[\xi_{0}](t)$ for $t < b$. In particular, for
$t=\xi_{0}$, this shows
$\mu[b](\xi_{0}) > \mu[\xi_{0}](\xi_{0}) = \mu(\xi_{0})$, which contradicts
the assumption $\mu[b] \leq \mu$ of Assumption~\ref{assump2}.


\paragraph{Repeat for remaining discontinuities.}
Before we proceed, let us summarize what we have established so far.
%
The times $\xi_{0} \in \halfopen{a}{t_{1}}$ and
$\tau_{1} \in \openhalf{t_{1}}{b}$ have been chosen such that
\begin{subequations}\label{eq:smoothing-requirements}
\begin{gather}
  \mu[\xi_{0}] \leq \mu \; \text{ for } \; t \in [\xi_{0}, \tau_{1}] , \\
  \dot{\mu}[\xi_{0}](\xi_{0}) = \dot{\mu}(\xi_{0}) \; \text{ and } \; \dot{\mu}[\xi_{0}](\tau_{1}) = \dot{\mu}(\tau_{1}) .
\end{gather}
\end{subequations}
Instead of $\xi_{0}$, it will be convenient later to choose $\xi_{1} := \max X$
as the representative of the unique connecting deceleration.
%
We can now use $\mu[\xi_{1}]_{[\xi_{1},\tau_{1}]}$ to replace $\mu$ at $[\xi_{1}, \tau_{1}]$ to
obtain a trajectory without the discontinuity at $t_{1}$. More precisely, we
define
\begin{align}
  \mu_{1}(t) =
  \begin{cases}
    \mu(t) &\text{ for } t \in [a, \xi_{1}] \cup [\tau_{1}, b] , \\
    \mu[\xi_{1}](t) &\text{ for } t \in [\xi_{1}, \tau_{1}] .
  \end{cases}
\end{align}
From the way we constructed $\mu[\xi_{1}]$, it follows
from~\eqref{eq:smoothing-requirements} that we have
$\mu_{1} \in \mathcal{P}[a,b]$, but without the discontinuity at $t_{1}$.
%
Observe that a single connecting deceleration may cover more than one
discontinuity, as illustrated in Figure~\ref{fig:multiple-discontinuities}.
%
Note that we must have $\dot{\mu}_{1}(a) = \dot{\mu}_{1}(b) = 1$.
%
Moreover, it is not difficult to see that $\mu_{1}$ must still satisfy
Assumption~\ref{assump2}, so that we can keep repeating the exact same process,
obtaining connecting decelerations
$(\xi_{2}, \tau_2), (\xi_{3}, \tau_{3}), \dots $ and the corresponding piecewise
trajectories $\mu_{2}, \mu_{3} \dots$ to remove any remaining discontinuities
until we end up with a smooth trajectory $\mu^{*} \in \mathcal{D}[a,b]$.
%
We emphasize again that $\dot{\mu}^{*}(a) = \dot{\mu}(a)$ and
$\dot{\mu}^{*}(b) = \dot{\mu}(b)$.

\begin{figure}[h]
  \centering
  \includegraphics[scale=1.1]{figures/motion/rough/multiple-discontinuities}
  \caption{Part of a piecewise trajectory $\mu$ on which a single connecting
    deceleration covers the two discontinuities at $t_{1}$ and $t_{2}$ at
    once.}%
  \label{fig:multiple-discontinuities}
\end{figure}


\subsubsection{Optimality after smoothing}

Let us return to the minimum boundary $\gamma$ defined
in~\eqref{eq:min-boundary}.
%
From Figure~\ref{fig:necessary-conditions} and the necessary conditions, it is
clear that $\gamma$ must satisfy $\gamma(a) = A$, $\gamma(b) = B$ and
\comment{I think this could be made more rigorous, but I think this is clear enough from the figure.}
$\dot{\gamma}(a) = \dot{\gamma}(b) = 1$, so whenever we have $\gamma \in \mathcal{D}[a, b]$, we
automatically have $\gamma \in D[a,b]$ so that $\gamma$ is already an optimal solution.
%
Otherwise, we perform the smoothing procedure presented above to obtain the
smoothed trajectory $\gamma^{*} \in \mathcal{D}[a,b]$.
%
The next lemma shows that it is still an upper boundary for any feasible
trajectory, which proves that $x^{*} := \gamma^{*}$ is an optimal solution to
the single vehicle problem.

\begin{lemma}\label{lemma:upperbound}
  Let $\mu \in \mathcal{P}[a,b]$ be a piecewise trajectory and let
  $\mu^{*} \in \mathcal{D}[a,b]$ denote the result after smoothing. All
  trajectories $x \in \mathcal{D}[a, b]$ that are such that
  $x \leq \mu$, must satisfy $x \leq \mu^{*}$.
  \comment{Is this argument clear enough?}
\end{lemma}
\begin{proof}
  Consider some interval $(\xi, \tau)$ where we introduced some connecting
  deceleration boundary. Suppose there exists some $t_{d} \in (\xi, \tau)$ such that
  $x(t_{d}) > \mu(t_{d})$. Because $x(\xi) \leq \mu(\xi)$, this means that $x$ must
  intersect $\mu$ at least once in $\halfopen{\xi}{t_{d}}$, so let
  $t_{c} := \sup \, \{ t \in \halfopen{\xi}{t_{d}} : x(t) = \mu(t) \}$ be the latest
  time of intersection such that $x \geq \mu$ on $[t_{c}, t_{d}]$. There must be some
  $t_{c} \in [t_{c}, t_{d}]$ such that $\dot{x}(t_{v}) > \dot{\mu}(t_{v})$, otherwise
  \begin{align*}
    x(t_{d}) = x(t_{c}) + \int_{t_{c}}^{t_{d}} \dot{x}(t) \diff t \leq \mu(t_{c}) + \int_{t_{c}}^{t_{d}} \dot{\mu}(t) \diff t = \mu(d_{t}) ,
  \end{align*}
  which contradicts our choice of $t_{d}$. Hence, for every
  $t \in [t_{v}, \tau]$, we have
  \begin{align*}
    \dot{x}(t) \geq \dot{x}(t_{v}) - \omega (t - t_{v}) > \dot{\mu}(t_{v}) - \omega(t - t_{v}) = \dot{\mu}(t) .
  \end{align*}
  It follows that $x(\tau) > \mu(\tau)$, which contradicts the assumption.
\end{proof}


\section{Computing optimal trajectories}

Recall the original trajectory optimization problem
\begin{subequations}
\begin{alignat}{3}
  &G(a,b) := \; &\max        \quad  & \sum_{i=1}^{N} \int_{a_{i}}^{b_{i}} x_{i}(t) \diff t , \\
  &             &\text{s.t.} \quad
  & x_{i} \in D[a_{i},b_{i}] & \text{ for each } i \in \{1, \dots, N\} , \\
  &&& x_{i} \leq x_{i-1} - L & \text{ for each } i \in \{2, \dots, N\} , \label{eq:follow-constraint}
\end{alignat}
\end{subequations}
where we use $a$ and $b$ to denote the vectors of arrival and departure times.
%
We now show how this problem decomposes into a sequence of instances of the
single vehicle problems.
%
Let the optimal solution of the single vehicle problem be denoted as
\begin{align}
  x^{*}(\alpha, \beta, \bar{x}) := \argmax_{x \in D[\alpha,\beta]} \int_{\alpha}^{\beta} x(t) \quad \text{ such that  } x \leq \bar{x}
\end{align}
and let $F(\alpha, \beta, \bar{x})$ denote the corresponding objective value.
%
Consider the safe following constraint~\eqref{eq:follow-constraint}. We show how
to model this as a boundary $\bar{x}_{i} \in \bar{D}[\bar{a}_{i}, \bar{b}_{i}]$
such that we can apply the single vehicle problem.
%
It is clear from Figure~\ref{fig:solution} that
inequality~\eqref{eq:follow-constraint} only applies on some subinterval
$I_{i} \subset [a_{i-1}, b_{i-1}]$.
%
More specifically, by defining
\begin{align}
  x_{i}^{-1}(p) := \inf \{ t: x_{i}(t) = p \} ,
\end{align}
it is easily seen that $I_{i} = [x_{i-1}^{-1}(L), b_{i}]$.
%
However, since $\dot{x}_{i-1}(b_{i-1}) = 1$, we can, roughly speaking, extend
\comment{Let's leave it like this for now. I think that it is not too difficult
  to see that this works by looking at the figure, so no need to introduce more
  notational burden.} the boundary until $b_{i-1} + L$, by defining
\begin{align}\label{eq:follow-boundary}
  \bar{x}_{i} :=
  \begin{cases}
    x_{i-1}(t) - L &\text{ for } t \in [x_{i-1}^{-1}(L), b_{i-1}] , \\
    t - b_{i-1} - L  &\text{ for } t \in [b_{i-1}, b_{i-1} + L] .
  \end{cases}
\end{align}

Now, it is then clear that the optimal trajectories $x_{i}$ can be
recursively computed as
\begin{subequations}
\begin{align}
  x_{1} &= x^{*}(a_{1}, b_{1}, \varnothing) , \\
  x_{i} &= x^{*}(a_{i}, b_{i}, \bar{x}_{i})  ,  \quad \text{ for } i \geq 2 ,
\end{align}
\end{subequations}
%
where we use the notation $\bar{x} = \varnothing$ to denote the single vehicle problem
without the boundary constraint. Alternatively, we could think about this as
having some $\bar{x} \in \bar{D}[\bar{a},\bar{b}]$ with very small $\bar{a} \ll a$
and $\bar{b} \ll b$.
%
The corresponding objective value is simply given by
\begin{align}
  G(a, b) = F(a_{1}, b_{1}, \varnothing) + \sum_{i=2}^{N} F(a_{i}, b_{i}, \bar{x}_{i}) .
\end{align}

\begin{figure}
  \centering
  \includegraphics[scale=1.0]{figures/motion/rough/solution}
  \caption{Optimal trajectories $x_{i}$ for three vehicles. The dotted blue
    trajectories between the little open circles illustrates the safe following
    constraints~\eqref{eq:follow-constraint}. The dotted blue trajectories between
    the solid dots are the following boundaries
    $\bar{x}_{2} \in \bar{D}[\bar{a}_{2}, \bar{b}_{2}]$ and
    $\bar{x}_{3} \in \bar{D}[\bar{a}_{3}, \bar{b}_{3}]$.}%
  \label{fig:solution}
\end{figure}


\subsection{Alternating trajectories}

Due to the recursive nature of the problem, we will see that optimal
trajectories possess a particularly simple structure, which enables a very
simple computation.

\begin{define}
  Let a trajectory $\gamma \in \mathcal{D}[a, b]$ be called \emph{alternating} if for all
  $t \in [a, b]$, we have
  \begin{align}
    \ddot{\gamma}(t) \in \{-\omega, 0, \bar{\omega}\} \quad \text{ and } \quad
    \ddot{\gamma}(t) = 0 \implies \dot{\gamma}(t) \in \{0, 1\}.
  \end{align}
\end{define}

We now argue that each vehicle's optimal trajectory $x_{i}$ is alternating.
First, consider $x_{1} = x^{*}(a_{1}, b_{1}, \varnothing)$, which is
constructed by joining $x^{1}[x_{1}]$ and $\hat{x}[x_{1}]$ together by smoothing. Observe that both boundaries are alternating by definition. Let
$\gamma_{1}(t) = \min\{x^{1}[x_{1}](t), \hat{x}[x_{1}](t) \}$ be the minimum boundary,
then it is clear that the smoothened $x_{1} = \gamma_{1}^{*}$ must also be
alternating, because we only added a part of deceleration at some interval
$[\xi, \tau]$, which clearly satisfies $\ddot{\gamma}_{1}^{*}(t) = -\omega$ for
$t \in [\xi,\tau]$.
%
Assume that $x_{i-1}$ is alternating, we can similarly argue that $x_{i}$ is
alternating. Again, let
$\gamma_{i}(t) = \min\{\bar{x}[x_{i-1}], \hat{x}[x_{i}](t), x^{1}[x_{i}](t)\}$ be the
minimum boundary. After adding the required decelerations for smoothing, it is
clear that $x_{i} = \gamma^{*}_{i}$ must also be alternating.

\begin{figure}
  \centering
  \includegraphics[scale=0.9]{figures/motion/tandem_trajectory}
  \caption{Example of an alternating vehicle trajectory with its defining time
    intervals. The particular shape of this trajectory is due to two leading
    vehicles, which causes the two start-stop \emph{bumps} around the times where these
    leading vehicles depart from the lane.}
  \label{fig:tandem_trajectory}
\end{figure}

Observe that an alternating trajectory $\gamma \in \mathcal{D}[a,b]$ can be described
as a sequence of four types of consecutive repeating phases, see
Figure~\ref{fig:tandem_trajectory} for an example.
%
In general, there exists a partition of $[a,b]$, denoted by
\begin{align*}
  a = t_{f1} \leq t_{d1} \leq t_{s1} \leq t_{a1} \leq t_{f2} \leq t_{d2} \leq t_{s2} \leq t_{a2} \leq \dots \leq t_{f,n+1} = b,
\end{align*}
such that we have the consecutive intervals
\begin{alignat*}{3}
  F_{i} &:= [t_{f,i}, t_{d,i}] \quad &\text{ (full speed), } \quad
  S_{i} &:= [t_{s,i}, t_{a,i}] \quad &\text{ (stopped), } \\
  D_{i} &:= [t_{d,i}, t_{s,i}] \quad &\text{ (deceleration), } \quad
  A_{i} &:= [t_{a,i}, t_{f,i+1}] \quad  &\text{ (acceleration), }
\end{alignat*}
%
such that on these intervals, $\gamma$ satisfies
%
\begin{alignat*}{4}
  &\dot{\gamma}(t) = 1 && \text{ for } t \in F_{i} , \quad \quad
  &&\dot{\gamma}(t) = 0 && \text{ for } t \in S_{i} ,\\
  &\ddot{\gamma}(t) = -\omega \quad && \text{ for } t \in D_{i} , \quad \quad
  &&\ddot{\gamma}(t) = \bar{\omega} \quad && \text{ for } t \in A_{i} .
\end{alignat*}
%
We will define parameterized functions $x^{1}$, $x^{-}$, $x^{0}$, $x^{+}$ to
describe $\gamma$ on each of these four types of intervals.
%
In the next section, we will show that this makes the smoothing procedure
particularly simple.

\subsection{Calculating smoothing times}

{\color{Navy}Derive $x^{+}$ similarly to how we derived $x^{-}$ when we
  introduced the deceleration boundary.}

It can be shown that smoothing introduces a part of deceleration $x^{-}$ only
between the four pairs of partial trajectories
\begin{align*}
    x^{+} \rightarrow x^{+} , \quad \quad
    x^{+} \rightarrow x^{0} , \quad \quad
    x^{1} \rightarrow x^{+} , \quad \quad
    x^{1} \rightarrow x^{0} .
\end{align*}
%

\begin{algorithm}
  \caption{Computing connecting deceleration for alternating trajectories.}
    \label{alg:connecting}
    \begin{algorithmic}
      \State Let $i$ such that $I_{i}$ is the latest such that $t_{1} < I_{i}$.
      \State Let $j$ such that $I_{j}$ is the earliest such that $t_{1} > I_{j}$.
    \end{algorithmic}
\end{algorithm}

\section{Feasibility as system of linear inequalities}

{\color{Navy} Show that the follow constraint $a_{i} \geq \bar{a}_{i}$ can be written in terms of $a_{i-1}$.}

We need to express the entry space constraint, condition (iv) in
Lemma~\ref{lemma:necessary-conditions}, in terms of the schedule times. Recall
that this conditions requires that
\begin{align}
  \bar{x}_{i} \geq \check{x}_{i} .
\end{align}
We will show that this condition can be rewritten in the form
\begin{align}
  a_{i} \geq \check{a}_{i}(a) ,
\end{align}
where $\check{a}_{i}(a,b)$ denotes some expression of the schedule times
$a_{1}, \dots, a_{n}$.

In conclusion, feasibility is expressed through the system of linear
inequalities
\begin{subequations}
\begin{align}
  b_{i} - a_{i} &\geq B - A &\text{ for all } i \in \{1, \dots, N\}, \\
  a_{i} &\geq  a_{i-1} - 1/\omega &\text{ for all } i \in \{2, \dots, N\} , \\
  b_{i} &\geq  b_{i-1} - 1/\omega &\text{ for all } i \in \{2, \dots, N\} , \\
  a_{i} &\leq \check{a}_{i}(a) &\text{ for all } i \in \{2, \dots, N \} .
\end{align}
\end{subequations}

\begin{figure}
  \centering
  \includegraphics[scale=1]{figures/motion/rough/bufferconstraint}
  \caption{Illustration of entry space constraint and the induced minimum entry
    time $\check{a}$.}%
  \label{fig:bufferconstraint}
\end{figure}

% \begin{figure}
%   \centering
%   \includegraphics[scale=1]{figures/motion/example1}
%   \caption{{\color{Navy} We include this old sketch here temporarily, which is
%       related to derive entry space constraint in terms of schedule times.}}%
% \end{figure}


\newpage
\appendix
\section{Miscellaneous}

\begin{lemma}\label{lemma:inf-continuous}
  Let $f : X \times Y \rightarrow \mathbb{R}$ be some continuous function. If
  $Y$ is compact, then the function $g : X \rightarrow \mathbb{R}$, defined as
  $g(x) = \inf \{ f(x,y) : y\in Y\}$, is also continuous.
\end{lemma}

\begin{lemma}\label{lemma:levelset}
  Let $f :\mathbb{R}^{n} \rightarrow \mathbb{R}^{m}$ be continuous and
  $y \in \mathbb{R}^{m}$, then the level set $N := f^{-1}(\{ y \})$ is a closed
  subset of $\mathbb{R}^{n}$.
\end{lemma}
\begin{proof}
  For any $y' \neq y$, there exists an open neighborhood $M(y')$ such that
  $y \notin M(y')$. The preimage $f^{-1}(M(y'))$ is open by continuity.
  Therefore, the complement
  $N^{c} = \{ x : f(x) \neq y \} = \cup_{y' \neq y} f^{-1}(\{y'\}) = \cup_{y' \neq y} f^{-1}(M(y'))$
  is open.
\end{proof}

\end{document}

% to enable the minted package
% Local Variables:
% TeX-command-extra-options: "-shell-escape"
% End:
