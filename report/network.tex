\documentclass[a4paper]{article}
\usepackage[margin=3.5cm]{geometry}
\usepackage{amsmath}
\usepackage{amssymb}
\usepackage[svgnames]{xcolor}
\usepackage{amsthm}
\usepackage{dsfont}
\usepackage{graphicx}
\usepackage{hyperref}
\usepackage{datetime}
\usepackage{outlines}
\usepackage{float}
\usepackage{booktabs}
\usepackage{enumitem}


\input{data/knitr_init.tex}

% code highlighting
\usepackage{minted}
\usepackage{xpatch}
\newminted[cminted]{python}{fontsize=\small}
\xpretocmd{\cminted}{\RecustomVerbatimEnvironment{Verbatim}{BVerbatim}{}}{}{}

% link coloring
%\hypersetup{
%    colorlinks,
%    linkcolor={red!80!black},
%    citecolor={green!60!black},
%    urlcolor={blue!80!black}
%}

% concatenation symbol (c.f. ++ in Haskell)
\newcommand\mdoubleplus{\mathbin{+\mkern-10mu+}}

% end of proof symbol
\newcommand{\newmarkedtheorem}[1]{%
  \newenvironment{#1}
    {\pushQED{\qed}\csname inner@#1\endcsname}
    {\popQED\csname endinner@#1\endcsname}%
  \newtheorem{inner@#1}%
}

\theoremstyle{definition}
%\newtheorem{eg}{Example}[section]
\newmarkedtheorem{eg}{Example}[section]
\newtheorem{observation}{Observation}[section]
\newtheorem{define}{Definition}[section]
\theoremstyle{plain}
\newtheorem{proposition}{Proposition}
\newtheorem{lemma}{Lemma}
\newtheorem{corollary}{Corollary}
\newtheorem{theorem}{Theorem}[section]
\newtheorem{assump}{Assumption}[section]
\newtheorem{remark}{Remark}[section]

\newdateformat{monthyeardate}{\monthname[\THEMONTH] \THEYEAR}

\author{Jeroen van Riel}
\date{\monthyeardate\today}
\title{Trajectory Optimization of Autonomous Vehicles in Networks of Intersections}

\begin{document}

\maketitle

\tableofcontents

\section{Trajectories in tandem of two intersections}

\begin{figure}
  \centering
  \makebox[\textwidth][c]{% wider than textwidth
    \includegraphics[width=1.0\textwidth]{figures/motion/tandem}
  }
  \caption{Tandem of two intersections $v$ and $w$ with lane of length $d(v,w)$.
    The grey rectangle represents some vehicle that just left intersection $v$.
    We will assume that a vehicle has maximum speed as long as it occupies an
    intersection, so it is only allowed to decelerate once it passes this
    position.}
  \label{fig:tandem}
\end{figure}

When considering multiple vehicle driving between intersection, we can no longer
ignore the issue of road capacity, because the fact that only a limited number
of vehicles can drive or wait at the same time on a lane between intersections
may cause network-wide effects.
%
The capacity of lanes between intersections
is intimately related to the trajectories of vehicles, which we first want to
understand better. We have been using an optimal control formulation with the
objective that keeps the vehicles as close as possible to the next intersection
at all times (\texttt{MotionSynthesize}). This problem can be solved using
direct transcription, which works well enough if we just want to simulate the
behavior of the system. However, we believe that it is possible to explicitly
formulate the optimal controller. We will explain how to compute trajectories
corresponding to those obtained by direct transcription, but without using time
discretization.

Before we turn to the general case of networks of intersection, we will first
investigate the trajectories of vehicles in a tandem of two intersections as
depicted in Figure~\ref{fig:tandem}. Let $v$ denote the left intersection and
$w$ the right intersection and assume that vehicles drive from left to right.
% We will sometimes refer to intersection $w$ as the downstream intersection.
Furthermore, we will call the road segment strictly between both intersection
areas the \textit{lane}. To facilitate the following discussion, we will use $p$
to denote the position of a vehicle to distinguish it from the state
$x = (p, v)$, which also includes the velocity, and we fix position $p=0$ at the
stop-line of intersection $w$. Let the length and width of a vehicle $i$ be
denoted by $L_{i}$ and $W_{i}$, respectively. We measure the position of a
vehicle at the front bumper.
%
We will make the following assumption, that allow us to easily derive explicit
expressions of these trajectories.

\begin{assump}
  \label{assump:same_geometry}
  All vehicles have the same length $L_{i} = L$ and width $W_{i} = W$. Lanes
  are exactly $W$ units wide and are axis-aligned, such that intersection are
  squares.
\end{assump}

\begin{assump}
  \label{assump:full_speed}
  Vehicles must drive at full speed when entering an intersection and keep
  driving at full speed as long as they occupy an intersection.
\end{assump}

Now assume that some vehicle is scheduled to exit $v$ at time $t_{0}$ and to
enter $w$ at some time $t_{f}$. Let $p_{0} = -d(v,w) + W$ denote the position of
the vehicle when it starts to exit $v$. Let $y(t)$ denote the position of the
vehicle that drives in front of the current one, assuming it exists. In order to
keep the vehicle as close to $w$ as possible at every time, while respecting
the double integrator vehicle dynamics $\dot{p} = v, \, \ddot{p} = u$, we can
generate a trajectory by solving the optimal control problem
\begin{equation}
  \label{eq:optimal_control}
  \begin{aligned}
  \max_{u}    \quad & \int_{t_{0}}^{t_{f}} p(t) dt \\
  \begin{alignedat}{2}\text{s.t.}\\ {}\end{alignedat} \quad &\begin{alignedat}{2}
                     0 \leq \; &v(t) \leq v_{\max} , \\
                     {-a_{\max}} \leq \; &u(t) \leq a_{\max} , \\
                    y(t) + L \leq \; & p(t) , \\
                    \end{alignedat} \\
                    &\begin{alignedat}[t]{2}
                    & p(t_{0}) = p_{0} , &&  p(t_{f}) = 0 , \\
                    & v(t_{0}) = v_{\max} , \;\; && v(t_{f}) = v_{\max} .
                    \end{alignedat}
  \end{aligned}
\end{equation}
%
This problem can be solved by using a direct transcription method. After
observing some example solutions, we believe that the optimal control should
switch between no acceleration and either full acceleration or full
deceleration, i.e., we have a control function $u(t) := \ddot{x}(t)$ that
satisfies $u(t) \in \{-a_{\max}, 0, a_{\max}\}$ and some sequence of alternating
deceleration and acceleration periods, represented by some sequence of disjoint
intervals
\begin{align*}
  (D_{0}, A_{1}, D_{1}, \dots, A_{n-1}, D_{n-1}, A_{n}) ,
\end{align*}
so that the optimal controller is given by
\begin{align*}
  u(t) = \begin{cases}
           {-a_{\max}} &\text{ if } t \in D_{k} \text{ for some } k , \\
           \phantom{-} a_{\max}   &\text{ if } t \in A_{k} \text{ for some } k , \\
           \phantom{-} \;\, 0 &\text{ otherwise. }
         \end{cases}
\end{align*}

\subsection{Single vehicle}

We can proof the above hypothesis and provide explicit expressions for $D_{k}$
and $A_{k}$ of the optimal trajectory whenever the safe following constraint
$y(t) \leq p(t)$ is never active and can essentially be ignored, either because
there is no vehicle ahead or it is sufficiently far away.
%
The proof uses a variant of the Pontryagin Maximum Principle, specifically
tailored to problems with state
constraints~\cite{hartlSurveyMaximumPrinciples1995}.

\newpage

\begin{figure}
  \centering
  \includegraphics[width=0.99\textwidth]{figures/motion/tandem_annotated}
  \caption{Tandem of intersections with distances used in the determination of
    the lane capacity and the waiting positions.}
  \label{fig:tandem_annotated}
\end{figure}

\subsection{Multiple vehicles}

By observing optimal trajectories generated using direct transcription, we
observe that there is a lot of structure in how the trajectory of a vehicle
influences the trajectories of later vehicles. In this section, we provide a
direct way to explicitly calculate trajectories, without proving that these
trajectories are always feasible and optimal. Instead, we verify correctness of
our method numerically by comparing our explicit trajectories to the the output
of direct transcription for various system parameters and crossing times.

We start by investigating the limit on the number of vehicles that can occupy
the lane at the same time. Imagine that vehicles enter the lane until it is full
and then only after all vehicles have come to a full stop, they start to leave
the lane by crossing $w$.
%
Without any additional constraints, it follows from the vehicle dynamics that
the time it takes to fully accelerate from rest to maximum velocity is given by
$d_{f} = v_{\max} / a_{\max}$ and the corresponding full acceleration trajectory
is given by
\begin{align*}
  p^{*}(t) = a_{\max} t^{2} / 2 \quad \text{ for } 0 \leq t \leq d_{f} .
\end{align*}
To keep the presentation simple, we assume that maximum deceleration is also
$a_{\max}$, so it also takes $d_{f}$ time to fully decelerate from maximum
velocity to rest.
%
Suppose that we want to design the tandem network such that at least $c(v,w)$
vehicles can enter and decelerate to some waiting position, from which it is
also possible to accelerate again to full speed before crossing $w$.
%
Vehicles are required to drive at full speed as long as they occupy
any intersection. Therefore, a vehicle crossing $v$ can only start decelerating
after $p(t) \geq p_{0} + L$, so the earliest position where a vehicle can come to a
stop is $p = p_{0} + L + p^{*}(d_{f})$.
%
Because vehicles need to gain maximum speed before reaching $w$,
the position closest to $w$ where a vehicle can wait is $- p^{*}(d_{f})$.
%
Hence, in order to accomodate for $c(v,w)$ waiting vehicles, the length of the
lane must satisfy
\begin{align*}
  d(v, w) \geq W + L + 2p^{*}(d_{f}) + (c(v,w) - 1) L ,
\end{align*}
as illustrated in Figure~\ref{fig:tandem_annotated}.
%
Conversely, given the lane length $d(v,w)$, the corresponding lane capacity is
given by
\begin{align*}
  c(v, w) = \texttt{floor}\left( \frac{d(v,w) - W - 2 p^{*}(d_{f})}{L} \right) ,
\end{align*}
where $\texttt{floor}(x)$ denotes the largest integer smaller than or equal to
$x$.

% emphasize fixed locations
It turns out that the fixed locations where vehicles wait in the above scenario
are helpful in describing the optimal trajectories, even when vehicles never
fully stop. We will denote these fixed \textit{waiting positions} as
\begin{align*}
  p_{k} = - p^{*}(d_{f}) - (c(v,w) - k) L,
\end{align*}
for $k = 1, \dots, c(v,w)$.
%
Furthermore, let $p_{d} = p_{1} - p^{*}(d_{f})$ denote the position from
which vehicles must decelerate in order to stop at the earliest waiting position
$p_{1}$, which is the farthest from the destination intersection.
%
Now consider a vehicle that moves from $p_{k}$ to the next waiting position
$p_{k+1}$, so it moves exactly distance $L$. We consider such a start-stop
movement, without considering any safe following constraints. By symmetry of the
control constraints, the vehicle moves the same distance during both
acceleration and deceleration. Furthermore, the vehicle needs to be at rest at
the start and end of such trajectory. Hence, it is clear that it takes the same
amount of time $d_{s}$ to accelerate and decelerate. We assume that
$d_{s} < d_{f}$, which ensures that $v_{\max}$ is never reached during the
start-stop movement, which is illustrated in Figure~\ref{fig:start-stop}. In this case, it is
clear that we must have $L = 2 p^{*}(d_{s})$, from which we derive that
$d_{s} = \sqrt{L / a_{\max}}$.

\subsubsection{Ad-hoc approach}
We will now present a method to calculate the trajectory of a vehicle based on
its crossing times at $v$ and $w$ and the trajectory of the vehicle ahead of it.
We start from a sequence of deceleration and acceleration intervals that are
possibly overlapping and then merge them in pairs from left to right until they
become disjoint.
%
Specifically, let
\begin{align*}
  t_{d} := t_{0} + \frac{p_{d} - p_{0}}{v_{\max}}
\end{align*}
be the start of the initial deceleration interval
$D_{0} := [t_{d}, t_{d} + d_{f}]$, which is exactly the moment the vehicle needs
to start decelerating in order to stop at the first waiting position $p_{1}$, so
at time $t_{d}$ the vehicle is at position $p_{d}$. Similarly, let
$t_{a} = t_{f} - d_{f}$ be the start of the final acceleration interval
$A_{f} := [t_{a}, t_{f}]$. Now for every $k = 1, \dots, c(v,w) - 1$, we
consider a pair of start-stop intervals
$S_{k} := (A_{k}, D_{k}) = ([t_{k}, t_{k} + d_{s}], [t_{k} + d_{s}, t_{k} + 2 d_{s}])$
at some starting time $t_{k}$, moving the vehicle from $p_{k}$ to $p_{k+1}$. We
first show how to merge these intervals such that a sequence of disjoint
intervals is obtained. Afterwards, we show how times $t_{k}$ follow from the
trajectory of the vehicle ahead.

% show how to merge, assuming $S_{n}$ are disjoint
First, we show how to merge $D_{0}$ and $S_{1}$.
% case 1: separate
When we have $t_{d} + d_{f} \leq t_{1}$, it is clear that both parts are already
completely disjoint, so we do not have to do anything further.
% case 2: merging
When $t_{d}$ and $t_{1}$ are closer than $d_{f}$, we need to merge $D_{0}$
and $A_{1}$. In this case, we observe that $D_{0}$ gets shorter
at the end by some $\epsilon$ and the acceleration part of $S_{1}$ gets shorter at the
beginning, also by the same amount $\epsilon$, because the velocities must match. More
precisely, we construct two new consecutive intervals
\begin{align*}
  D_{0}' &= [t_{1} + 2 \epsilon - d_{f}, \; t_{1} + \epsilon] , \\
  A_{1}' &= [t_{1} + \epsilon,   \; t_{1} + d_{s}],  \\
  D_{1}  &= [t_{1} + d_{s},      \; t_{1} + 2 d_{s}] .
\end{align*}
%
We now determine $\epsilon$ as a function of $t_{1} - t_{d}$.
%
Because $D_{0}'$ and $A_{1}'$ are both $\epsilon$ shorter, the total distance
that is traversed is now $2 p^{*}(\epsilon)$ shorter. This means that the start
of $D_{0}'$ should be $2 p^{*}(\epsilon) / v_{\max}$ later than $t_{d}$, which
gives equation
\begin{align*}
  t_{d} + \frac{2 p^{*}(\epsilon)}{v_{\max}}  = t_{1} + 2\epsilon - d_{f} ,
\end{align*}
which is quadratic in $\epsilon$. Because we must have $\epsilon < d_{f}$, we need the
smallest solution
\begin{align*}
  \epsilon = d_{f} - \sqrt{d_{f} (t_{1} - t_{d})} .
\end{align*}
% case 3: merged
When we keep increasing $t_{d}$ while keeping $t_{1}$ fixed, we observe that
eventually part $A_{1}$ will completely disappear and $D_{0}$ and $D_{1}$ will
become a single interval, which is easily seen to happen when $\epsilon \geq d_{s}$.
Equivalently, this happens when $t_{d}$ is such that
$t_{d} \geq t_{1} - d_{f} + 2 d_{s} - 2p^{*}(d_{s}) / v_{\max}$.

% merge start-stops (AD-AD)
We now show how two start-stop parts have to be merged if they overlap. Consider
two start-stop parts $A_{k},D_{k}$ and $A_{k+1},D_{k+1}$ and suppose that
$t_{k} + 2 d_{s} < t_{k+1}$ such that both parts have to be merged. The parts
$D_{k}$ and $A_{k+1}$ merge by removing $\epsilon$ on both sides, similarly as
above. However, this causes the $D_{k},A_{k+1},D_{k+1}$ parts to shift up.
Therefore, $A_{k}$ and $D_{k}$ each need to be lengthened at the side where they
meet by some $\delta$ to match this.
%
Hence, it turns out we have to use the intervals
\begin{align*}
  A_{k}' &= [t_{k}, t_{k} + d_{s} + \delta] , \\
  D_{k}' &= [t_{k} + d_{s} + \delta, t_{k+1} + \epsilon] , \\
  A_{k+1}' &= [t_{k+1} + \epsilon, t_{k+1} + d_{s} ] , \\
  D_{k+1} &= [t_{k+1}, t_{k+1} + d_{s}] .
\end{align*}
%
We have that $\epsilon$ and $\delta$ need to satisfy
\begin{align*}
  \begin{cases}
  2\delta + 2\epsilon = t_{k+1} - t_{k} , \\
  2p^{*}(d_{s}+\delta) - 2p^{*}(d_{s} - \epsilon) = L .
  \end{cases}
\end{align*}
Solving this system of equations yields
\begin{align*}
  \delta &= \frac{L / a_{\max}}{t_{k+1} - t_{k}} - d_{s} + \frac{t_{k+1} - t_{k}}{4} , \\
  \epsilon &= (t_{k+1} - t_{k}) / 2 - \delta .
\end{align*}
%

Finally, we consider the merge with the final acceleration bang.

Note that the above types of merging are enough to process the whole sequence of
bangs, because when the first merge of $D_{0}$ and $A_{1},D_{1}$ results in a
single $D_{1}'$ and there is a next $A_{2},D_{2}$, we are again in the first
situation.

% show how $S_{n}$ follow from predecessor
We now show how $t_{k}$ follow from the trajectory of the preceding vehicle.

\begin{figure}
  \centering
  \includegraphics[scale=1.0]{figures/motion/start_stop_trajectory}
  \caption{Shape of some start-stop trajectory of a single isolate vehicle
    moving forward from some current waiting position $p_{k}$ to the next
    $p_{k+1}$.}
  \label{fig:start-stop}
\end{figure}

\begin{figure}
  \centering
  \includegraphics[width=0.8\textwidth]{figures/motion/tandem_trajectory}
  \caption{Sketch of vehicle trajectory in tandem with all the full start-stop
    parts unmerged.}
  \label{fig:tandem_trajectory}
\end{figure}

% \begin{figure}
%   \centering
%   \includegraphics[width=0.99\textwidth]{figures/motion/merge}
%   \caption{Sketch how the initial deceleration merges with the first start-stop phase.}
%   \label{fig:merge}
% \end{figure}

\subsubsection{Schedule time approach}

\newpage
\section{Trajectories in networks}

\begin{figure}
  \centering
  \includegraphics[width=0.55\textwidth]{figures/network/grid_example.png}
  \caption{Illustration of some grid-like network of intersections with vehicles
    drawn as grey rectangles. There are five vehicle routes: two from east to
    west and three from south to north. Turning at intersections is not
    allowed.}\label{fig:network_illustration}
\end{figure}

We now extend the single intersection model to a network of intersections
without turning routes, illustrated in Figure~\ref{fig:network_illustration}.
% network definition
We define a directed graph $(\bar{V},E)$ with nodes $\bar{V}$ and arcs $E$,
representing the possible paths that vehicles can follow. Nodes with only
outgoing arcs are \textit{entrypoints} and nodes with only incoming arcs are \textit{exitpoints}.
Let $V$ be the set of \textit{intersections}, which are all the nodes with
in-degree at least two.
%
Let $d(v, w)$ denote the distance between nodes $v$ and $w$.
%
For each route index $r \in \mathcal{R}$, we let
\begin{align*}
  \bar{V}_{r} = (v_{r}(0), v_{r}(1), \dots, v_{r}(m_{r}), v_{r}(m_{r+1}))
\end{align*}
be the path that vehicles $i \in \mathcal{N}_{r}$ follow through the network. We
require that the first node $v_{r}(0)$ is an entrypoint and that the last node
$v_{r}(m_{r+1})$ is an exitpoint and we write
\begin{align*}
  V_{r} = \bar{V}_{r} \setminus \{ v_{r}(0), \, v_{r}(m_{r+1}) \}
\end{align*}
to denote the path restricted to intersections. We say that some $(v, w) \in E$
is on path $V_{r}$ whenever $v$ and $w$ are two consecutive nodes on the path
and we write $E_{r}$ to denote the set of all these edges. We require that
routes can only overlap at nodes by making the following assumption.

\begin{assump}\label{assump:disjoint_routes}
  Every arc $(v,w) \in E$ is part of at most one route $V_{r}$.
\end{assump}

We start by considering networks in which all roads are axis-aligned such that
intersections always involve perpendicular lanes and where routes are such that
no turning is required. For each $v \in V_{r}$ define the conflict zone
$\mathcal{E}_{r}(v) = (b_{r}(v), e_{r}(v))$ and consider the union
\begin{align*}
  \mathcal{E}_{r} = \bigcup_{v \in V_{r}} \mathcal{E}_{r}(v)
\end{align*}
corresponding to the positions of vehicles $i \in \mathcal{N}_{r}$ for which it
occupies an intersection on its path $V_{r}$.
%
By reading $\mathcal{E}_{i} \equiv \mathcal{E}_{r}$ for $r(i) = r$, the single
intersection problem naturally extends to the network case. Like before, the
resulting problem can be numerically solved by a direct transcription method.

\subsection{General decomposition}
The general two-stage decomposition for the single intersection extends rather
naturally to the present model. Let for each pair $(i,v)$ of some vehicle
$i \in \mathcal{N}$ and an intersection $v \in V_{r(i)}$ along its route, let
\begin{align*}
\inf \{ t: x_{i}(t) \in \mathcal{E}_{r}(v) \} \;\; \text{ and } \; \sup \{ t: x_{i}(t) \in \mathcal{E}_{r}(v) \}
\end{align*}
be the crossing time and exit time, which we denote by $y(i,v)$ and
$y(i,v) + \sigma(i, v)$, respectively.
%
Instead of a single set of conflicts, we now define for each intersection
$v \in V$ in the network the set of conflict pairs
\begin{align*}
\mathcal{D}^{v} = \{ \{i,j\} \subset \mathcal{N} : r(i) \neq r(j), v \in V_{r(i)} \cap V_{r(j)} \} .
\end{align*}
Now the two-stage approach is to solve
\begin{align*}
  \min_{y,\sigma} \;\; & \sum_{r \in \mathcal{R}} F(y_{r}, \sigma_{r}) \\
  \text{ s.t. } & y(i,v) + \sigma(i,v) \leq y(j,v) \text{ or }  \\
                & y(j,v) + \sigma(j,v) \leq y(i,v) , & \text{ for all } \{i,j\} \in \mathcal{D}^{v} \text{ and } v \in V, \\
  & (y_{r}, \sigma_{r}) \in \mathcal{S}_{r} , \quad & \text{ for all } r \in \mathcal{R} ,
\end{align*}
%
where $F(y_{r}, \sigma_{r})$ and $\mathcal{S}_{r}$ are the value function and
set of feasible parameters, respectively, of the parametric trajectory
optimization problems
%
\begin{align*}
  F(y_{r}, \sigma_{r}) = \min_{x_{r}} & \; \sum_{r \in \mathcal{R}} J(x_{i}) \\
  \text{ s.t. } & x_{i}(t) \in D_{i}(s_{i,0}) , \quad & \text{ for } i \in \mathcal{N}_{r} , \\
  & x_{i}(y(i,v)) = b_{r}(v) , \quad & \text{ for } v \in V_{r} , i \in \mathcal{N}_{r} , \\
  & x_{i}(y(i,v) + \sigma(i,v)) = e_{r}(v) , \quad & \text{ for } v \in V_{r} , i \in \mathcal{N}_{r} , \\
  & x_{i}(t) - x_{j}(t) \geq L , \quad & \text{ for } (i, j) \in \mathcal{C} \cap \mathcal{N}_{r} ,
\end{align*}
where we again use subscript $r$ to group variables according to their associated route.


\subsection{Decomposition for delay objective}

Suppose we use use the crossing at the last intersection as performance measure, by defining the
objective function as
\begin{align*}
  J(x_{i}) = \inf \{ t: x_{i}(t) \in \mathcal{E}_{r}(v_{r}(m_{r}))\} .
\end{align*}
%
We show how to reduce the resulting problem to a scheduling problem, like we did
in the single intersection case.
%
We will again assume Assumption~\ref{assump:same_geometry} and
Assumption~\ref{assump:full_speed}, so vehicles will always cross intersections
at full speed, and all vehicles share the same geometry. Hence, the occupation
time $\sigma \equiv \sigma(i,v)$ is the same for all vehicles and intersections. For this
reason, we will write the shorthand $y_{r} \in \mathcal{S}_{r}$, because $\sigma_{r}$
is no longer a free variable.

As a consequence of Assumption~\ref{assump:same_geometry} and Assumption~\ref{assump:full_speed},
each lower-level trajectory optimization problem for a given route
$r \in \mathcal{R}$ decomposes into a sequence of problems, each corresponding to
two consecutive intersection along $V_{r}$.
%
This means that $y_{r} \in \mathcal{S}_{r}$ is equivalent to
$y_{(v,w)} \in \mathcal{S}_{(v,w)}$ for each $(v,w) \in E_{r}$, where
$y_{(v,w)}$ denotes the vector of all variables $y(i, v)$ and $y(i, w)$ for all
$i \in \mathcal{N}_{r}$ and $\mathcal{S}_{(v,w)}$ denotes the set of values of $y_{(v,w)}$ for which a feasible trajectory part can be found.
%
Hence, we will now focus on a tandem of two intersections and investigate the
trajectories of vehicles in this with the goal of stating sufficient conditions
for $y_{(v,w)} \in \mathcal{S}_{(v,w)}$.

\subsection{Crossing time scheduling}

\begin{itemize}
  \renewcommand\labelitemi{--}
{\color{gray}
% job-shop disjunctive graph
\item Introduce general job-shop problem and disjunctive graph, see Figure~\ref{fig:disjunctive_graph}.
% define travel constraints
\item Introduce travel constraints and its disjunctive graph arcs.
% define buffer constraints
\item Introduce buffer constraints and its disjunctive graph arcs.

% branch-and-bound
\item Formulate MILP problem and investigate how the solving time scales with network
size in terms of number of intersections and number of vehicles in the network.
Do the single intersection cutting planes still hold? Are there any obvious
cutting planes?
}
\end{itemize}

Lower bounds $\beta(i, v)$ on the crossing times of vehicles, where
$i = (r, k) \in \mathcal{N}$ is some vehicle index and $v \in V_{r}$.

\begin{figure}
  \centering
  \includegraphics[scale=1]{figures/network/disjunctive_graph}
  \caption{Empty disjunctive graph belonging to a tandem of two intersections,
    labeled $v$ and $w$. There are three routes $\mathcal{R} = \{0, 1, 2\}$. On
    route 0, four vehicle are arriving, the other two routes have 3 arrivals
    each. Conjunctive arcs are drawn as from left to right, travel arcs are
    drawn from top to bottom and disjunctive arcs are drawn as dotted lines. Two
    buffer constraints are drawn as diagonal arcs, corresponding to a capacity
    of 2 vehicles for the lane between both intersections.}
  \label{fig:disjunctive_graph}
\end{figure}

\newpage
\section{Constructive heuristics}

Like in the single intersection case, we are going to model optimal solutions
using autoregressive models.
%
In the single intersection case, we argued that each schedule is uniquely
defined by its route order $\eta$. Generalizing this to the case of multiple
intersections, we see that a schedule is uniquely defined by the collection of
route orders $\eta^{v}$ at each intersection $v \in V$. Instead of working with
such a set of sequences, we will intertwine the sequences to obtain a single
\textit{crossing sequence} $\eta$, which is a sequence of pairs $(r, v)$ of a
route $r \in \mathcal{R}$ at some intersection $v \in V_{r}$ and we will refer
to such a pair as a \textit{crossing}. Of course, this crossing sequence can be
constructed in many different ways, because it does not matter in which order
the intersections are considered, to which we will refer as the
\textit{intersection visit order}.
% \begin{align*}
%   | \{ t : \nu_{t} = v \} | = |\eta^{v}| \quad \text{ for each } v \in V ,
% \end{align*}
Given some problem instance $s$, we will consider autoregressive models of the form
\begin{align}
  \label{eq:autoregressive}
  p(\eta | s) =  \prod_{t=1}^{N} p(\eta_{t}| s, \eta_{1:t-1}) .
\end{align}

These models can also be understood in terms of a step-by-step schedule
construction process that transitions from a partial schedule state
$s_{t} = (s, \eta_{1:t})$ to the next state $s_{t+1}$ by selecting some
\textit{action} $\eta_{t} = (r_{t}, v_{t})$. We say a crossing is pending when
it still has unscheduled vehicles. Similarly, we say an intersection is pending
when some of its crossings are still pending. Therefore, the set of \textit{valid actions}
$\mathcal{A}(s_{t})$ at some intermediate state $s_{t}$ is exactly the set of
pending crossings.
%
% multiple correct actions
We again emphasize that multiple sequences of actions lead to the same schedule,
because the order in which intersections are considered does not matter for the
final schedule, which is illustrated in Figure~\ref{fig:solution_equivalence}.
%
The models that we study will mostly be parameterized as some function of the
disjunctive graph representation $G_{t}$ of partial schedule $s_{t}$, so an
alternative way of writing~\eqref{eq:autoregressive} that emphasizes this is
\begin{align*}
  p(\eta = ((r_{1}, v_{1}), \dots, (r_{N}, v_{N})) \; | \; s) = \prod_{t=1}^{N} p(r_{t}, v_{t} | G_{t-1}) .
\end{align*}

Instead of modeling the probability distribution $p(r_{t}, v_{t} | G_{t-1})$
over crossings, we apply the chain rule to factorize this joint probability as
\begin{align*}
  p(r_{t}, v_{t} | G_{t-1}) = p(r_{t} | v_{t} , G_{t-1}) p (v_{t} | G_{t-1}) .
\end{align*}
This allows us to consider models that assume some fixed intersection visit
order $\mu$, such that instead of a single next action, the model should model a
set of actions, one for each intersection.
%
We will show that $\mu$ has a considerable impact on the final schedule quality.
However, it is not clear a priori which $\mu$ is best, so we will propose some
possible ways to define $p(v_{t} | G_{t-1})$.
% random
First of all, the simple \textit{random} strategy would be to sample some
intersection with pending crossings at each step.
% ``boundary'' (or ``exhaustive'')
In the \textit{boundary} strategy, we keep visiting the same intersection until
it is done (when it has no pending crossings anymore), then move to some next
intersection. When the network of intersections is free of cycles, we could for
example follow some topological order. We use the term ``boundary'' because this
strategy produces trajectories along the boundary of the grid in
Figure~\ref{fig:solution_equivalence}.
% ``alternate''
In the \textit{alternating} strategy, we keep alternating between intersection
to keep the number of scheduled vehicles balanced among them. This produces
trajectories that can be understood as being close to the ``diagonal'' of the
grid in Figure~\ref{fig:solution_equivalence}.

It is straightforward to extend the threshold rule to networks of
intersections, when assuming a fixed intersection visit order. Each time some
next intersection is visited, we apply the single intersection threshold rule to
pick the next route. This is straightforward to do, because we can just consider
the disjunctive subgraph induced by the nodes belonging to that intersection to
arrive at the single intersection case.
%
Furthermore, the definition of the threshold rule itself does not depend on the
network of intersections. This is a desirable property, because it allows us to
tune the threshold on small networks and then apply it on larger ones.


\subsection{Neural constructive heuristic}

We will now propose a neural network parameterization of
$p(r_{t} | v_{t}, G_{t-1})$ and train it based on optimal schedules in a
supervised learning setting. The model can be best understood as solving a
so-called multi-label classification problem, because it needs to provide a
distribution over routes at every intersection.
%
The training data set $\mathcal{X}$ consists of pairs
$(G_{t-1}, (r_{t}, v_{t}))$, to which we refer to as \textit{state-action} pairs
to draw the parallel with the terminology used in reinforcement learning.
%
To obtain these pairs, we sample a collection of problem instances, which are
solved to optimality using a MILP solver. For each optimal schedule, we compute
the corresponding optimal route order $\eta^{v}$ for each intersection.
%
From these, we can construct $\mathcal{X}$ in different ways.
% sample a single intersection order for each instance
For example, for each solved instance, we can randomly select some intersection
visit order $\mu$, which fixes the crossing order $\eta$. We can then replay
this sequence of actions step-by-step to obtain the corresponding sequence of
state-action pairs.
% multiple intersection order per instance
The model might become more robust when training on multiple samples of
intersections visit orders per instance.
% fixed intersection order
Alternatively, we can consider one of the fixed intersection visit orders
described at the start of this section (random, exhaustively, alternating).
% lookahead
Furthermore, combined with one of the above strategies, we can also employ some
kind of fixed lookahead procedure, as illustrated in
Figure~\ref{fig:state_action_collection}.
% inference
At inference time, use the same intersection visit order as during training and
at each step we select greedily select $r_{t}$.

We will now describe how the model is parameterized based on recurrent
embeddings of the sequences of crossing time lower bounds for unscheduled
vehicles at each crossing, exactly the same as how we defined the
\textit{horizons} in the single intersection model. Each such horizon is
embedded using an Elman RNN. These embeddings are then fed through a feedforward
network to produce a logit for each crossing, see Figure~\ref{fig:rnn_model}.

% silent package loading


\begin{knitrout}
\definecolor{shadecolor}{rgb}{0.969, 0.969, 0.969}\color{fgcolor}\begin{table}

\caption{\label{tab:unnamed-chunk-2}Comparison of threshold heuristics.}
\centering
\begin{tabular}[t]{cc|c|ccc|c|ccc|c|ccc|c|c}
\toprule
n & size & MILP & $\tau = 0$  (gap)\\
\midrule
5 & 2x1 & 57.27 & 65.27 (11.45\%)\\
5 & 3x1 & 57.67 & 68.34 (15.44\%)\\
5 & 3x2 & 57.35 & 69.17 (18.32\%)\\
\bottomrule
\end{tabular}
\end{table}

\begin{table}

\caption{\label{tab:unnamed-chunk-2}Comparison of neural heuristics.}
\centering
\begin{tabular}[t]{cc|c|c|ccc|c|c|ccc|c|c|ccc|c|c|ccc|c|c|c}
\toprule
n & size & random (gap) & exhaustive (gap) & alternate (gap)\\
\midrule
5 & 2x1 & 58.98 (2.98\%) & 58.90 (2.84\%) & 58.15 (1.53\%)\\
5 & 3x1 & 60.03 (4.11\%) & 59.42 (3.04\%) & 58.74 (1.86\%)\\
5 & 3x2 & 61.28 (6.86\%) & 60.49 (5.47\%) & 58.62 (2.22\%)\\
\bottomrule
\end{tabular}
\end{table}

\end{knitrout}


\begin{figure}
  \centering
  \includegraphics[scale=1]{figures/network/rnn_model}
  \caption{Illustration of model with RNN encoding of horizon at each crossing.
    The disjunctive graph nodes whose lower bounds are part of the current
    horizons are indicated with rectangles. The RNN embeddings (open circles)
    are fed through a final feedforward network to produce a logit (dots) for
    each crossing.}
  \label{fig:rnn_model}
\end{figure}

\begin{figure}
  \centering
  \includegraphics[width=0.4\textwidth]{figures/network/solution_equivalence}
  \caption{When we fix the ordering of actions locally at every intersection,
    the global order of actions is not unique, because the local sequences may
    be merged in any way. Suppose we have a tandem of two intersections and the
    horizontal arrows correspond to taking the next local action at intersection
    1, the vertical arrows correspond to taking next local action at
    intersection 2. The figure shows two possible global orders. Of course, the
    exact global order does not matter for the final schedule. However, it might
    be that trajectories near the boundary of the grid are harder to learn from
    data than trajectories that stay closer to the diagonal. The intuition is
    that we need to ``look into the future'' more to learn the former, while in
    the latter trajectories, progress in the two intersections is more balanced.}
  \label{fig:solution_equivalence}
\end{figure}

\begin{figure}
  \centering
  \includegraphics[scale=1.5]{figures/network/state_action_collection}
  \caption{Possible strategy for collecting state-action pairs: one action
    look-ahead. At every state $s$ with a set of possible actions
    $\mathcal{A}(s)$, we add all state-action pairs
    $\{ (s, a) : a \in \mathcal{A}(s) \}$, but we pick a single action $a^{*}$
    to move the automaton further. In the figure, the state-action pairs are
    depicted as solid arrows. The encircled dots are the states that are
    actually visited. Observe that this procedure can be naturally generalized
    to a look-ahead of arbitrary depth.}
  \label{fig:state_action_collection}
\end{figure}

\subsection{Performance evaluation}

For some given problem instance, let $N$ denote the length of a crossing
sequence, so it is also the total number of vehicle-intersection pairs $(i, v)$
that need to be scheduled. To allow a fair comparison of methods accross
instances of different sizes, both in terms of the number of vehicles as the
size of the network, we define the objective of schedule $y$ as the average
delay per vehicle-intersection pair, as given by
\begin{align*}
  \textrm{obj}(y) = \frac{1}{N} \sum_{i \in \mathcal{N}} \sum_{v \in V_{r(i)}} y(i, v) - a(i, v) .
\end{align*}


{\color{Navy} We will use the performance of threshold heuristic as a baseline
  when evaluating heuristics based on a neural architecture. Furthermore, it
  also provides a \textit{baseline} during reinforcement learning, reducing the
  variance of the REINFORCE estimator. }



\newpage
\subsection{Network-agnostic architecture}

We now focus on models that generalize across network sizes. In particular, we
are interested in models that can be learned on relatively small networks (for
which exact solutions might be tractable) and can then be applied to larger
networks.

We use a GIN to compute an embedding for each node, which is then fed through an
MLP and softmax to produce a probability over nodes. The GNN computes node
embeddings, which are mapped to a score for each node. We compute the softmax
over the scores of the nodes and then compute the negative log likelihood loss
for backpropagation.
%
Like in Zhang et al., each action corresponds to a unique node, encoding the
operations that is dispatched next. However, we only really need to provide a
route-intersection pair, but how to exploit this in the policy model?
%
At this point, during the collection of observation-action pairs, we copy the
whole disjunctive graph for each state. Alternatively, we could use some sort of
masking for non-final states.



\bibliography{references}
\bibliographystyle{ieeetr}

\end{document}

% to enable the minted package
% Local Variables:
% TeX-command-extra-options: "-shell-escape"
% End:
