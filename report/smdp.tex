\documentclass{article}
\usepackage{amsmath}
\usepackage{amssymb}
\usepackage{xcolor}
\usepackage{amsthm}
\usepackage{graphicx}
\usepackage{hyperref}
\usepackage{datetime}
\usepackage{outlines}
\usepackage[round]{natbib}   % omit 'round' option if you prefer square brackets

\newdateformat{monthyeardate}{\monthname[\THEMONTH] \THEYEAR}

\newcommand{\newmarkedtheorem}[1]{%
  \newenvironment{#1}
    {\pushQED{\qed}\csname inner@#1\endcsname}
    {\popQED\csname endinner@#1\endcsname}%
  \newtheorem{inner@#1}%
}

\theoremstyle{definition}
%\newtheorem{eg}{Example}[section]
\newmarkedtheorem{eg}{Example}[section]
\newtheorem{observation}{Observation}[section]
\newtheorem{define}{Definition}[section]
\theoremstyle{plain}
\newtheorem{proposition}{Proposition}[section]
\newtheorem{theorem}{Theorem}[section]
\newtheorem{assump}{Assumption}[section]
\newtheorem{remark}{Remark}[section]

\author{Jeroen van Riel}
\date{\monthyeardate\today}
\title{Two-queue polling system with switching}

\begin{document}


\section{Model Formulation}
\label{sec:model}

Consider a system in which two infinite-capacity queues are attended by a single
server. Customers arrive to lane $i$ as a Poisson process with arrival rate
$\lambda_{i}$. The server switches between both queues, so let the location of the
server be denoted by $u \in \{1,2\}$. Customers from queue $i$ can only be served
whenever $u = i$. We assume that switching and processing (service) times are
deterministic, denoted by $s$ and $p$, respectively, and that these actions
cannot be preempted, once started. We model this system as a semi-Markov
decision process, starting with a natural formulation and then showing how to
derive a simplified version with a discrete (countable) state space.

\subsection{Natural formulation}

The actions are \textit{Process} (serve) a customer, \textit{Switch} to the other queue and \textit{Idle},
so the action space is $\mathcal{A} = \{ P, S, I \}$. Let the number of customers
in queue $i$ be denoted by $x_{i}$. We will use the vector notation
$x = (x_{1}, x_{2})$ and define $e_{1} = (1, 0), e_{2} = (0, 1)$. Each state is
identified by the tuple $(u, \rho, \sigma, x)$, where $\rho$ denotes the remaining service
time and $\sigma$ denotes the remaining switch time. The corresponding state space is
\begin{align}
  \mathcal{S} = \{1, 2\} \times \mathbb{R}^{+} \times \mathbb{R}^{+} \times \mathbb{N}^{+} \times \mathbb{N}^{+} ,
\end{align}
where $\mathbb{R}^{+}$ and $\mathbb{N}^{+}$ denote the nonnegative reals and
integers, respectively.

We will first discuss the different types of state transitions that are
possible, without exactly specifying the corresponding transition probabilities.
For each transition, $s$ denotes the current state, $a$ denotes the action, $s'$
denotes the next state and $\tau$ denotes the sojourn time. Let $\bar{u}$ denote
the other queue. To support a concise description of the transitions, we set
$\rho=p$ and $\sigma=s$ whenever the server is not processing or switching. Not every
action is always available to the decision maker. Whenever $\rho < p$, the only
allowed action is $P$ and similarly, whenever $\sigma < s$, the only allowed action
is $S$, to model the non-preemptive nature of these actions. Note that the
server can only process a customer when the current queue is nonempty, hence
action $P$ is only allowed when $x_{u} > 0$. We have the following types of
transitions:
%
\begin{subequations}
\begin{itemize}
  \item service completion ($x_u > 0$)
  \begin{align}
    \begin{split}
    s &= (u, \rho, s, x), \; a=P, \\ s' &= (u, p, s, x - e_{u}), \; \tau = \rho ,
    \end{split}
  \end{align}

  \item arrival to queue $i$ when serving ($x_{u} > 0, \tau < \rho$)
  \begin{align}
    \label{eq:arrival_serve}
    \begin{split}
    s &= (u, \rho, s, x), \; a=P, \\ s' &= (u, \rho - \tau, s, x + e_{i}) ,
    \end{split}
  \end{align}

  \item switch completion
  \begin{align}
    \begin{split}
    s &= (u, p, \sigma, x), \; a=S, \\ s' &= (\bar{u}, p, s, x), \; \tau = \sigma ,
    \end{split}
  \end{align}

  \item arrival to queue $i$ when switching ($\tau < \sigma$)
  \begin{align}
    \label{eq:arrival_switch}
    \begin{split}
    s &= (u, p, \sigma, x), \; a=S, \\ s' &= (u, p, \sigma - \tau, x + e_{i}) ,
    \end{split}
  \end{align}

  \item arrival to queue $i$ when idling
  \begin{align}
    \begin{split}
    s &= (u, p, s, x), \; a=I, \\ s' &= (u, p, s, x + e_{i}) .
    \end{split}
  \end{align}
\end{itemize}
\end{subequations}


We will now discuss the rewards associated to the system. We assume that holding
costs are levied at a constant unit rate for each customer in the system. Let
$\Upsilon_{m}$ denote the sojourn time of the $m$-th transition and let
$T_{m} = \sum_{j=1}^{m} \Upsilon_{j}$ denote the completion time of the $m$-th transition.
The continuous reward rate process $\{ r(t) : t \in \mathbb{R}^{+} \}$ is
defined by
\begin{align}
  r(t) = - (x_{1} + x_{2}) \; \text{ for } t \in [ T_{m}, T_{m+1} ) ,
\end{align}
where $m \geq 1$.

For optimization, we will consider an infinite horizon with the total discounted
reward criterion
\begin{align}
  \label{eq:criterion}
  \phi_{\beta} = \mathbb{E} \left[ \int_{0}^{\infty} e^{-\beta t} r(t) dt \right] ,
\end{align}
where $\beta > 0$ is known as the \textit{discount factor}.


\subsection{Reducing the state space}
\label{sec:reduced-model}

Although the above formulation is appealing because of its interpretability, we
will now discuss how to obtain an equivalent model with a countable state space.
We assumed that the serve and switch actions cannot be preempted. Therefore, no
actual decision is necessary whenever an arrival occurs while the server is
either serving or switching, so we skip these states and continue to the state
where an actual decision must be made. This means that we disregard states with
$\rho < p$ or $\sigma < s$ and the corresponding transitions
in~\eqref{eq:arrival_serve} and~\eqref{eq:arrival_switch}, so that states can
now be represented as $(u, x)$, with state space
\begin{align}
  \mathcal{S} = \{ 1, 2 \} \times \mathbb{N}^{+} \times \mathbb{N}^{+} .
\end{align}
Instead of modeling single arrivals, the transitions must now enable any number
of arrivals to happen during a serve or switch action. In the following, let
$N(t) = (N_{1}(t), N_{2}(t))$ denote a random vector with $N_{i}(t)$ being the
number of arrivals to lane $i$ that happen during a time interval of length $t$,
which is distributed as a $\text{Pois}(\lambda_i t)$ random variable. The reduced
model now has the following types of transitions:
%
\begin{subequations}
\begin{itemize}
  \item service completion ($x_u > 0$)
  \begin{align}
    \begin{split}
    s &= (u, x), \; a=P, \\ s' &= (u, x - e_{u} + N(p)), \; \tau = p ,
    \end{split}
  \end{align}

  \item switch completion
  \begin{align}
    \begin{split}
    s &= (u, x), \; a=S, \\ s' &= (\bar{u}, x + N(s)), \; \tau = s ,
    \end{split}
  \end{align}

  \item arrival to queue $i$ when idling
  \begin{align}
    \begin{split}
    s &= (u, x), \; a=I, \\ s' &= (u, x + e_{i}) .
    \end{split}
  \end{align}
\end{itemize}
\end{subequations}

Note that the reward process $r(t)$ is still valid for the reduced model, but
its value may change between our new states, which is not convenient if we want
to apply existing methods that assume otherwise. Therefore, we will redefine the
rewards in terms of an immediate reward (also called \textit{lump sum} by some
authors) that is earned at the start of the transition.
%
Let $\{S_{m} : m \in \mathbb{N}^{+} \}$ denote the embedded state Markov chain
of the natural model. Let $N \subset \mathbb{N}^{+}$ denote the indices of the
states that are not skipped so that $\{S_{n} : n \in N \}$ is the embedded state
Markov chain of the reduced model. Let $\bar{n}$ denote the predecessor of
$n$ in $N$, then the sojourn times in the reduced model are given by
\begin{align}
  \Upsilon'_{n} = \sum_{m=\bar{n} + 1}^{n} \Upsilon_{m} .
\end{align}
%
By collecting the total reward over all skipped transitions, the immediate
reward of the $n$-th transition in the reduced model is given by
\begin{align}
  R_{n} = \int_{0}^{\Upsilon'_{n}} e^{- \beta s} r(T_{\bar{n}} + s) ds .
\end{align}
Because we are now dealing with point masses at $T_{n}$ for $n \in N \setminus \{ 0 \}$, define
\begin{align}
        R(t) = \sum_{n \in N : T_{\bar{n}} \leq t} R_{n} .
\end{align}
Hence, we have
\begin{align}
  \begin{split}
  \int_{0}^{\infty} e^{- \beta t} d R(t) &= \sum_{n \in N \setminus \{ 0 \}} e^{- \beta T_{\bar{n}}} R_{n} \\
   &= \sum_{n \in N \setminus \{ 0 \}} \int_{T_{\bar{n}}}^{T_{n}} e^{- \beta t} r(t) dt \\
   &= \int_{0}^{\infty} e^{-\beta t} r(t) dt ,
  %&= \sum_{n \in N} e^{- \beta T_{\bar{n}}} \int_{0}^{\Upsilon'_{n}} e^{- \beta t} r(T_{\bar{n}} + t) dt  \\
  \end{split}
\end{align}
which shows that the discounted reward criterion~\eqref{eq:criterion} is exactly the same in the
reduced model.

Because we will focus on the reduced model from now on, we let
$\{ S_{m} : m \in \mathbb{N}^{+} \}$ refer to the states of the reduced model
and let $\Upsilon_{m}$ and $R_{m}$ denote the sojourn time and reward,
respectively, of the $m$-th transition. Using
$T_{m} = \sum_{j=1}^{m} \Upsilon_{j}$, the objective is given by
\begin{align}
  \phi_{\beta} = \mathbb{E} \left[ \sum_{m=1}^{\infty} e^{- \beta T_{m-1}} R_{m} \right] .
\end{align}


\section{Optimal Policies}

We will use the following notation for a general SMDP. We say that a policy is
\textit{pure}, whenever the action at transition $m$ is a function of the
current state only. Let $\pi : \mathcal{S} \rightarrow \mathcal{A}$ denote some
pure policy. To avoid cluttering the notation, we ignore the dependency of the
process on $\pi$. Let $\{ S_{m} : m \in \mathbb{N}^{+} \}$ denote the embedded
state process, then at each epoch $m$, some action $A_{m} = \pi(S_{m})$ is
taken, upon which reward $R_{m+1}$ is observed and the state changes to
$S_{m+1}$ after $\Upsilon_{m+1}$ sojourn time. The transition probabilities are
defined by
\begin{align}
  P_{xay} = \mathbb{P}(S_{m+1} = y \; | \; S_{m} = x, A_{m} = a) .
\end{align}
Conditional on the event that the next state is $y$, we consider the joint
distribution of the reward and sojourn time
\begin{align}
  F_{xay}(r, \tau) = \mathbb{P}(R_{m+1} \leq r, \Upsilon_{m+1} \leq \tau \; | \; S_{m} = x, A_{m} = a, S_{m+1} = y ) .
\end{align}
Note that we cannot simply consider the marginals in our case, because the
reward and sojourn time are not independent. The continuous-time process
$\{ S(t), A(t), R(t) : t \geq 0 \}$ where $S(t)$ is the state of the process at
time $t$, $A(t)$ is the action taken at time $t$ and $R(t)$ is the cumulative
reward up to time $t$, is referred to as the SMDP.

When introducing the discounted total reward criterion in~\eqref{eq:criterion},
we did not make explicit the dependency on the initial state $z$. Therefore, we
define
\begin{align}
  \phi_{\beta}^{\pi}(z) = \mathbb{E} \left[ \int_{0}^{\infty} e^{- \beta t} dR(t) \; \big| \; S(0) = z \right] .
\end{align}
%
For the discounted total reward criterion,
it can be shown that there exists a pure policy $\pi^{*}$ that is optimal among
all policies, i.e.,
\begin{align}
  \phi_{\beta}^{\pi^{*}}(z) = \sup_{\pi} \phi_{\beta}^{\pi}(z) ,
\end{align}
for some fixed initial state $z \in \mathcal{S}$. Well-known methods of finding
optimal policies include policy iteration, value iteration and using linear
programming.

\subsection{Exhaustive service}

A slightly more general SMDP model as introduced in Section~\ref{sec:model} has
been analyzed by~\cite{hofriOptimalControlTwo1987}; instead of deterministic
service and switch time, they consider random service times with the same
distribution for both queues and switch times with possibly different
distributions for both directions. They show (Proposition 2.1) that an optimal
policy must always perform exhaustive service, which means that the server
always continues serving the current queue when it is nonempty.

In terms of our notation, their result implies that if $S_{m} = (u, x)$
satisfies $x_{u} > 0$, then any optimal policy $\pi^{*}$ satisfies
$\pi^{*}(S_{m}) = P$. Therefore, we skip such states, because there is no actual
choice to be made. Observe that each state now satisfies $x_{u} = 0$, so we only
have to keep track of the number of customers in the other queue $x_{\bar{u}}$.
The resulting reduced model that we will consider from here on has action space
$\mathcal{A} = \{ S, I \}$ and state space
\begin{align}
  \mathcal{S} = \{1,2\} \times \mathbb{N}^{+} .
\end{align}

\subsection{Double-threshold policy}

For state $S_{m} = (u, x)$, we will use $g(S_{m}) = g(u, x_{1}, x_{2})$ to
simplify notation. A policy $g : \mathcal{S} \rightarrow \mathcal{A}$ is said to
be a \textit{double-threshold policy} if it performs exhaustive service, as
defined above, and it satisfies
\begin{subequations}
\begin{align}
  g(1, 0, x_{2}) = \begin{cases}
                     I, & x_{2} < m_{2} , \\
                     S, & x_{2} \geq m_{2} ,
                    \end{cases} \\
  g(1, x_{1}, 0) = \begin{cases}
                     I, & x_{1} < m_{1} , \\
                     S, & x_{1} \geq m_{1} ,
                    \end{cases}
\end{align}
\end{subequations}
for nonnegative threshold values $m_{1}$ and $m_{2}$. It is conjectured
by~\cite{hofriOptimalControlTwo1987} that there exists an optimal
double-threshold policy (Theorem 3.4). Their proof is incomplete, but the
missing part is reduced to two concavity properties of the optimization
objective as function of the initial state. Our goal is to test their conjecture
empirically by using model-free reinforcement learning to find an optimal policy
(and estimate the true objective function to verify the concavity properties).


\subsection{Symmetric system}
\label{sec:symmetric_system}

Whenever $\lambda_{1} = \lambda_{2} = \lambda$, we say that the system is
\textit{symmetric}. We show there is a symmetric optimal policy $\pi^{*}$ for
such system, in the sense that $\pi^{*}(1,x) = \pi^{*}(2,x)$. Assuming the
conjecture of the previous subsection to be true, this means that there exists
an optimal double-threshold policy with $m_{1} = m_{2} = m$. Furthermore, this
result implies that, whenever we have a symmetric system, we may equivalently
consider the SMDP with state space $\mathcal{S} = \mathbb{N}^{+}$.

For any pure policy $\pi$, let $\pi'$ be defined as
$\pi'(u, x) = \pi(\bar{u}, x)$ for any state $(u, x) \in \mathcal{S}$. From the
symmetry of the system, it follows that for every pure policy $\pi$, we have
\begin{align}
  \phi_{\beta}^{\pi}(u, x) = \phi_{\beta}^{\pi'}(\bar{u}, x) ,
\end{align}
for every $(u, x) \in \mathcal{S}$. Let $\pi^{*}$ denote an optimal pure policy
and define
\begin{align}
g(u, x) = \pi^{*}(1, x) ,
\end{align}
then we have
\begin{align}
  \phi_{\beta}^{g}(u, x) = \phi_{\beta}^{\pi^{*}}(1, x) = \sup_{\pi} \phi_{\beta}^{\pi} (1, x) = \sup_{\pi} \phi_{\beta}^{\pi'} (2, x) ,
\end{align}
for every $(u, x) \in \mathcal{S}$, which shows that $g$ is an optimal pure
policy that satisfies $g(u, x) = g(\bar{u}, x)$.



\subsection{Curse of modeling}

In principle, it is possible to explicitly derive the transition probabilities
of the model. However, this might turn out to be difficult in practice, a
problem which is sometimes refered to as the curse of modeling. To avoid
explicitly expressing the dynamics of the system, one could use simulation to
obtain estimates of the transition probabilities, which can then be used in
stochastic dynamic programming like value iteration or policy iteration to
obtain an optimal policy for the approximation model.
%
Alternatively, model-free reinforcement learning can be used, which circumvents
the need to compute transition probabilities at all, which we will discuss next.

\subsection{Q-learning}

Let the Q-value of taking action $a$ in state $s$ be defined as
\begin{align}
  Q(s, a) = Q_{\beta}^{\pi}(s, a) = \mathbb{E} \left[ \phi_{\beta}^{\pi}(s) \; \big| \;  A(0) = a \right] .
\end{align}
%
Recall that the Q-update in classical Q-learning for MDPs with the discounted
total reward criterion~(see for instance Section 6.5 in \cite{suttonReinforcementLearningIntroduction2018}) is given by
\begin{align}
  Q(S_{m}, A_{m}) \leftarrow (1 - \alpha) Q(S_{m}, A_{m}) + \alpha[ R_{m+1} + \gamma \max_{a} Q(S_{m+1}, a) ] .
\end{align}
%
In order to extend this method to be used with SMDPs, we need to take into
account the fact that sojourn times affect the amount of discounting
\citep{gosaviSimulationBasedOptimizationParametric2015}, so that the Q-update is
now given by
\begin{align}
  Q(S_{m}, A_{m}) \leftarrow (1 - \alpha) Q(S_{m}, A_{m}) + \alpha[ R_{m+1} + e^{- \beta \Upsilon_{m+1}} \max_{a} Q(S_{m+1}, a) ] .
\end{align}


\subsection{Results}

We use Q-learning for a symmetric system with $\lambda = 1/3, p = 1$ and
$s = 10$. As expected, the algorithm converges to a threshold policy with $m=5$.
However, after some manual experimentation with hyperparameters, it seems that
we need a lot of samples in order to converge to this policy.

\subsection{Approximating the SMDP}

Instead of directly learning from samples using a model-free method like
Q-learning, we can also estimate the transition probabilities and the
distributions of $R_{n}$ and $\Upsilon_{n}$ from samples, obtaining an explicit
specification of an approximation SMDP.\@ The latter can then be solved exactly
using dynamic programming, e.g., using policy iteration.


\section{Waiting}

The action space that we have considered up till now is not complete in some
sense. More precisely, given some feasible schedule $y$, there should be some
policy $\pi^{y}$ that depends in some (very complicated) way on $y$, such that
executing $\pi^{y}$ over the offline instance produces $y$. What we are missing
is an action that allows the server to wait at the current queue for some finite
time $\delta$. Therefore, we define the additional waiting action $I(\delta)$
with $\delta \in \{ \mathbb{R}^{+}, \infty \}$, such that
$\mathcal{A} = \{ P, S, I(\infty) \} \cup \{ I(\delta) : \delta > 0 \}$, where
$I(\infty)$ is equivalent to the original $I$ action. The $I(\delta)$ action
causes the server to idle for $\delta$ time units at the current queue and then
switch, unless another arrival happens at the current queue before $\delta$
time.

From the examples that we discussed in the offline scheduling setting, we
conjecture that the optimal policy for the two-queue polling model must exploit
some kind of waiting strategy. Therefore, we are going to evaluate
double-threshold policies with waiting. In order to simplify the investigation,
we consider a symmetric system in which the arrival rates $\lambda_{i} = \lambda$ are the
same, because it can be shown that this implies $m_{i} = m$, using a similar
argument as in Section~\ref{sec:symmetric_system}.

\begin{proposition}
  For Poisson arrivals, $\delta$-idling with $\delta < \infty$ is always suboptimal.
\end{proposition}


\section{General Arrivals}

We would like to drop the assumption of Poisson arrivals. Because the memoryless
property does not hold anymore, we need to keep track of the time since the last
arrival in the states. Therefore, states now need to be represented by the tuple
$(u, x, \nu)$, where $\nu = (\nu_{1}, \nu_{2})$ and $\nu_{i}$ is the time since
the last arrival to queue $i$. We are now back in the situation of an
uncountable state space, which requires us to apply discretization before being
able to apply Q-learning.


\section{Discussion}

One might aks whether it is possible to transform the SMDP of
Section~\ref{sec:reduced-model} to an equivalent MDP, which might be beneficial
in practice, because most off-the-shelf computational methods need some slight
adapatation to be suitable for SMDPs. This transformation is possible, but would
require us to introduce a continuous part to the state space, because we need to
keep track of how much time the last transition took.

The idea of skipping states as we used in Section~\ref{sec:reduced-model} is
related to the use of \textit{options}~\citep{suttonMDPsSemiMDPsFramework1999},
where the basic idea is that we can also choose \textit{options}, which are a
sort of sub-policies that specify how to take actions over multiple steps, in
addition to ordinary or \textit{primitive} actions. The original paper considers
an MDP as the underlying model, but as they point out in a footnote (page 3),
this idea can be extended to SMDPs (I have not yet searched the literature to
see if someone has done this).

Most literature on reinforcement learning only focuses on expected quantities
and ignores higher moments of the underlying distributions. In our model, the
reward is not deterministic given the current state, action and next state
$(s, a, s')$. I am not sure how this affects the convergence of Q-learning. The
topic of distributional reinforcement
learning~\citep{bellemareDistributionalPerspectiveReinforcement2017,bdr2023}
seems to be concerned with similar questions.


\bibliography{references}
\bibliographystyle{plainnat}

\end{document}
