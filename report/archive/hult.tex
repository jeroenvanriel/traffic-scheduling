\documentclass{article}
\usepackage{amsmath}
\usepackage{amssymb}
\usepackage{xcolor}
\usepackage{amsthm}
\usepackage{dsfont}
\usepackage{graphicx}
\usepackage{hyperref}
\usepackage{datetime}
\usepackage{outlines}
\usepackage[round]{natbib}

\usepackage{matlab-prettifier}

\newdateformat{monthyeardate}{\monthname[\THEMONTH] \THEYEAR}

\newcommand{\newmarkedtheorem}[1]{%
  \newenvironment{#1}
    {\pushQED{\qed}\csname inner@#1\endcsname}
    {\popQED\csname endinner@#1\endcsname}%
  \newtheorem{inner@#1}%
}

\theoremstyle{definition}
%\newtheorem{eg}{Example}[section]
\newmarkedtheorem{eg}{Example}[section]
\newtheorem{observation}{Observation}[section]
\newtheorem{define}{Definition}[section]
\theoremstyle{plain}
\newtheorem{proposition}{Proposition}[section]
\newtheorem{theorem}{Theorem}[section]
\newtheorem{assump}{Assumption}[section]
\newtheorem{remark}{Remark}[section]

\author{Jeroen van Riel}
\date{\monthyeardate\today}
\title{}

\begin{document}


\subsection*{Vehicle dynamics}

In control theory, it is common to model motion dynamics of a system in terms of
a state vector $x(t) \in \mathbb{R}^{n}$ and a control input vector
$u(t) \in \mathbb{R}^{m}$, which result in a scalar position $y(t)$ via the
equations
\begin{subequations}
\begin{align}
  \dot{x}(t) &= A x(t) + B u(t) , \\
  y(t) &= C x(t) .
\end{align}
\end{subequations}
%
Furthermore, it is common to restrict the state and control trajectories by
imposing linear constraints
\begin{subequations}\label{eq:control_constraints}
\begin{align}
  G x(t) \leq b , \\
  F u(t) \leq d .
\end{align}
\end{subequations}

In the discussion that follows, each vehicle is modeled as a \textit{double
  integrator}, with $x(t) = (p(t), v(t))$, where $p(t)$ and $v(t)$ are the
scalar position along a predefined path and corresponding velocity,
respectively. The three matrices are chosen such that
\begin{subequations}
\begin{align}
  \dot{x}(t) &= \begin{pmatrix} 0 & 1 \\ 0 & 0 \end{pmatrix} x(t) + \begin{pmatrix} 0 \\ 1 \end{pmatrix} u(t), \\
  y(t) &= \begin{pmatrix} 1 & 0 \end{pmatrix} x(t),
\end{align}
\end{subequations}
which may simply be rewritten as
\begin{align}
  \label{eq:motion_dynamics}
  \dot{p}(t) = v(t) , \quad
  \dot{v}(t) = u(t) , \quad
  y(t) = p(t) ,
\end{align}
where we recognize that the control input $u(t)$ corresponds directly to the
acceleration of the vehicle.
%
Furthermore, the constraints~\eqref{eq:control_constraints} are chosen such that
the acceleration is bound from above and below, so
\begin{align}
  \label{eq:bounded_acceleration}
  \underline{u} \leq u(t) \leq \overline{u} .
\end{align}
For technical reasons, it is assumed the system is \textit{strongly output
  monotone}, defined as
\begin{align}
  \label{eq:output_monotone}
  \dot{y}(t) \geq \epsilon ,
\end{align}
for some $\epsilon > 0$, which means that a vehicle cannot stop or reverse, but
can move at an arbitrarily low speed.


\subsection*{Intersection model}

Consider an intersection with $L$ lanes. We define the index set
\begin{align}
  \mathcal{I} = \{ (l, k) : k \in \{1, \dots, n_{l}\}, \; l \in \{1, \dots L\}\} ,
\end{align}
where $n_{l}$ denotes the number of vehicles of lane $l$. To
further help with notation, given vehicle index $i = (r,s) \in \mathcal{I}$, we
define $l(i) = r$ and $k(i) = s$.

We assume that the position $p_{i}(t)$ of some vehicle $i \in \mathcal{I}$
corresponds to the physical front of the vehicle. In order to model collision
avoidance, we say that a vehicle \textit{occupies the intersection} whenever
$p_{i}(t) \in [L_{i}, H_{i}] = \mathcal{E}_{i}$. The collision avoidance
constraints are then given by
\begin{align}
  (p_{i}(t), p_{j}(t)) \notin \mathcal{E}_{i} \times \mathcal{E}_{j},
\end{align}
for all $t$ and for all pairs of indices $i, j \in \mathcal{I}$ with
$l(i) \neq l(j)$, which we collect in the set $\mathcal{D}$.
%
Furthermore, in order to model a safe distance between vehicles on the same
lane, we require that
\begin{align}
  p_{i}(t) - p_{j}(t) \geq P ,
\end{align}
for all $t$ and all pairs of indices $i, j \in \mathcal{I}$ such that
$l(i) = l(j), \; k(i) + 1 = k(j)$, which we collect in $\mathcal{C}$.
%
Let $D_{i}(x_{i,0})$ denote the set of feasible trajectories
$x_{i}(t) = (p_{i}(t), v_{i}(t), u_{i}(t))$ given some initial state $x_{i,0}$
and satisfying the vehicle dynamics given by
equations~\eqref{eq:motion_dynamics},~\eqref{eq:bounded_acceleration}
and~\eqref{eq:output_monotone}. Given some performance criterion
\begin{align}
  J(x_{i}) = \int_{0}^{t_{f}} \Lambda(x_{i}(t)) dt ,
\end{align}
where $t_{f}$ denotes the final time, the coordination problem is formulated as
\begin{subequations}\label{eq:full_problem}
\begin{align}
  \min_{\mathbf{x}(t)} \quad & \sum_{i \in \mathcal{I}} J(x_{i}) \\
  \text{s.t.} \quad  & x_{i} \in D_{i}(x_{i,0}) , &\text{for all } i \in \mathcal{I} , \\
                & (p_{i}(t), p_{j}(t))  \notin \mathcal{E}_{i} \times \mathcal{E}_{j} , &\text{for all } (i,j) \in \mathcal{D} , \\
                & p_{i}(t) - p_{j}(t) \geq P, &\text{for all } (i,j) \in \mathcal{C} ,
\end{align}
\end{subequations}
where $\mathbf{x}(t) = [\, x_{i}(t) : i \in \mathcal{I} \,]$.

\subsection*{Exact solution}

We discretize problem~\eqref{eq:full_problem} on a uniform time grid.
Let $K$ denote the number of discrete time steps and let $\Delta t$ denote the time step size.
%
We use the forward Euler integration scheme as follows
\begin{subequations}
\begin{align}
  p_{i}(t + \Delta t) = p_{i}(t) + v_{i}(t) \Delta t , \\
  v_{i}(t + \Delta t) = v_{i}(t) + u_{i}(t) \Delta t .
\end{align}
\end{subequations}
The disjunctive constraints are formulated using the big-M technique by the constraints
\begin{subequations}
\begin{align}
  p_{i}(t) \leq L + \delta_{i}(t) M , \\
  H - \gamma_{i}(t) M \leq p_{i}(t) , \\
  \delta_{i}(t) + \delta_{j}(t) + \gamma_{i}(t) + \gamma_{j}(t) \leq 3 ,
\end{align}
\end{subequations}
where $\delta_{i}(t), \gamma_{i}(t) \in \{ 0, 1 \}$ for all $i \in \mathcal{I}$
and $M$ is a sufficiently large number.
%
Finally, the follow constraints can simply be enforced at each time step for
each pair of consecutive vehicles in $\mathcal{C}$.

\subsection*{Decomposition}

The \textit{entry} and \textit{exit} times of vehicle $i$ are given,
respectively, by
\begin{align}
  \tau_{i} = t : p_{i}(t) = L_{i} , \quad \xi_{i} = t : p_{i}(t) = H_{i} .
\end{align}

Define the optimization problem
\begin{subequations}
\begin{align}
  \label{eq:F_opt}
  F_{i}(\tau_{i}, \xi_{i}) = \min_{x_{i}(t)} & \; J(x_{i}) \\
                              \text{s.t. } & x_{i} \in D_{i}(x_{i,0}), \\
                                           & p_{i}(\tau_{i}) = L, \\
                                           & p_{i}(\xi_{i}) = H.
\end{align}
\end{subequations}

Given some vehicle $i = (l, k)$, define
\begin{align}
  \mathcal{N}(i, n) = \{ j \in \mathcal{I} : l(j) = l(i), \; k(j) \in \{ k(i), \dots, k(i) + n - 1\}\} ,
\end{align}
to which refer to as the \textit{lane successors} of vehicle $i$. Now we
generalize problem~\eqref{eq:F_opt} by defining
\begin{subequations}
\begin{align}
  F(i, \tau, \xi, n) = \min_{x_{j}(t) : j \in \mathcal{N}(i, n)} & \; \sum_{j \in \mathcal{N}(i, n)} J(x_{j}) \\
                          \text{s.t. } & x_{j} \in D_{j}(x_{j,0}), \text{ for all  } j \in \mathcal{N}(i, n) , \\
                                       & p_{i}(\tau) = L, \\
                                       & p_{j}(\xi) = H, \text{ for } j = (l(i), k(i) + n - 1) , \\
                                       & p_{a}(t) - p_{b}(t) \geq P, \text{ for all } (a, b) \in \mathcal{N}(i, n)^{2} \cap \mathcal{C} ,
\end{align}
\end{subequations}
%
such that $F_{i}(\tau_{i}, \xi_{i}) = F(i, \tau_{i}, \xi_{i}, 1)$.

The idea is to generalize scheduling of single vehicle time slots to scheduling of time slots for platoons of consecutive vehicles.

\end{document}
