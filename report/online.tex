\documentclass[a4paper]{article}
\usepackage{amsmath}
\usepackage{amssymb}
\usepackage{xcolor}
\usepackage{amsthm}
\usepackage{dsfont}
\usepackage{graphicx}
\usepackage{hyperref}
\usepackage{datetime}
\usepackage{outlines}
\usepackage{float}
\usepackage{booktabs}
\usepackage{enumitem}


% code highlighting
\usepackage{minted}
\usepackage{xpatch}
\newminted[cminted]{python}{fontsize=\small}
\xpretocmd{\cminted}{\RecustomVerbatimEnvironment{Verbatim}{BVerbatim}{}}{}{}

% link coloring
%\hypersetup{
%    colorlinks,
%    linkcolor={red!80!black},
%    citecolor={green!60!black},
%    urlcolor={blue!80!black}
%}

% concatenation symbol (c.f. ++ in Haskell)
\newcommand\mdoubleplus{\mathbin{+\mkern-10mu+}}

% end of proof symbol
\newcommand{\newmarkedtheorem}[1]{%
  \newenvironment{#1}
    {\pushQED{\qed}\csname inner@#1\endcsname}
    {\popQED\csname endinner@#1\endcsname}%
  \newtheorem{inner@#1}%
}

\theoremstyle{definition}
%\newtheorem{eg}{Example}[section]
\newmarkedtheorem{eg}{Example}[section]
\newtheorem{observation}{Observation}[section]
\newtheorem{define}{Definition}[section]
\theoremstyle{plain}
\newtheorem{proposition}{Proposition}
\newtheorem{lemma}{Lemma}
\newtheorem{corollary}{Corollary}
\newtheorem{theorem}{Theorem}[section]
\newtheorem{assump}{Assumption}[section]
\newtheorem{remark}{Remark}[section]

\newdateformat{monthyeardate}{\monthname[\THEMONTH] \THEYEAR}

\author{Jeroen van Riel}
\date{\monthyeardate\today}
\title{}

\begin{document}

\section*{Online control with re-optimization}

The model assumes that routes are edge-disjoint and that vehicles drive at full
speed over the intersection area.
%
We add vehicles to the network in order of increasing arrival time. After a new
arrival happened, we calculate the current optimal set of crossing times
relative to the current time. Next, we simulate this schedule forward in time
until the next arrival.
%
The re-optimization problem is obtained by generalizing the network MILP to
incorporate vehicle positions at a certain time.

\subsection*{Instantaneous acceleration}

We only include a crossing time variable $y(i, v)$ in the MILP when vehicle $i$
still needs to cross intersection $v$.
% definitions
Let $x(i)$ denote the current position of vehicle $i$, relative to the first
node of its route. Let $(u(i), v(i))$ denote the current edge of vehicle $i$.
Let $p(i)$ denote the \textit{edge position}, which is the position on the
current edge, so relative to $u(i)$.

\vspace{1em}
\noindent
For conjunctions, we need to consider three cases:
\begin{itemize}
 \item both vehicles still need to cross the intersection
 \item vehicle 2 is occupies the intersection
 \item both vehicles crossed the intersection
\end{itemize}

\noindent
For disjunctions, we need to consider four cases:
\begin{itemize}
 \item both vehicles still need to cross the intersection
 \item vehicle 1 still occupies the intersection
 \item vehicle 2 still occupies the intersection
 \item both vehicles crossed the intersection
\end{itemize}


\noindent
Distance constraints:
\begin{itemize}
 \item current edge, take into account current edge position
 \item unvisited edges, use edge length
\end{itemize}

\noindent
Capacity constraints:
\begin{itemize}
 \item not yet crossed
 \item crossed
\end{itemize}


\subsection*{Bounded acceleration}

This problem is more restrictive, because some decisions are fixed due to the
inability of vehicles to stop immediately.



\bibliography{references}
\bibliographystyle{ieeetr}


\end{document}

% to enable the minted package
% Local Variables:
% TeX-command-extra-options: "-shell-escape"
% End:
