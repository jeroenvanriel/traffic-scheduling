\documentclass{article}
\usepackage{amsmath}

\title{Traffic Scheduling}
\author{Jeroen van Riel}
\date{September 2023}

\begin{document}

\maketitle

\section{Job Shop Scheduling}
We have $m$ machines and $n$ jobs. The basic job shop scheduling problem may be
formulated as the following MIP:
\begin{align*}
  \text{minimize} \max_{ij} C_{ij} \\
  y_{ij} + p_{ij} &\leq y_{kj}  & \text{ for all } (i,j) \xrightarrow{} (k,j) \in A \\
  y_{ij} + p_{ij} &\leq C_{\text{max}} & \text{ for all } (i,j) \in N \\
  y_{il} + p_{il} &\leq  y_{ij} \text{ or } y_{ij} + p_{ij} \leq y_{il}  & \text{ for all } (i,l) \text{ and } (i,j), i =1, \dots,m \\
  y_{ij} &\geq 0 & \text{ for all } (i,j) \in N
\end{align*}


\section{Traffic Scheduling}
We now consider a situation with different job types, setup
(switch-over/clearance) times between different job types and minimal delay
between operations that belong to the same job (to model a vehicle traveling
along an edge).
%
\begin{itemize}
  \item jobs on the same edge (same ``type'') must respect the release date order
  \item operations belonging to the same job, traveling from intersection $i$ to $k$ must be at least $t_{ik}$ apart
  \item switch-over time is required between jobs of different type on same machine (this probably required binary variables, or disjunction in any case)
\end{itemize}
%
Preserve ordering of vehicles on each incoming edge, because we assume vehicles do not overtake each other:
\begin{align*}
  r_{j} < r_{l} \implies (i,j) \rightarrow{} (i,l) \text{ for all } i = 1,\dots,m .
\end{align*}
%
\begin{align*}
  \text{minimize} \max_{ij} C_{ij} \\
  y_{ij} + p_{ij} + t_{ik} &\leq y_{kj}  & \text{ for all } (i,j) \xrightarrow{} (k,j) \in A \\
  y_{ij} + p_{ij} &\leq C_{\text{max}} & \text{ for all } (i,j) \in N \\
  y_{il} + p_{il} &\leq  y_{ij} \text{ or } y_{ij} + p_{ij} \leq y_{il}  & \text{ for all } (i,l) \text{ and } (i,j), i =1, \dots,m \\
  y_{ij} &\geq 0 & \text{ for all } (i,j) \in N \\
  %
\end{align*}
\end{document}
