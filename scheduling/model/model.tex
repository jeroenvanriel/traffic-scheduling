\documentclass{article}
\usepackage{amsmath}

\title{Traffic Scheduling}
\author{Jeroen van Riel}
\date{Oktober 2023}

\begin{document}

\maketitle

\section{Job Shop Scheduling}

This section shortly introduces the classical \textit{job shop} scheduling
problem. Assume there are $m$ machines and $n$ jobs. A job $j$ consists of
exactly $m$ operations, one for each of the machines. We let $(i,j)$ denote the
operation of job $j$ that needs to be processed on machine $i$. The time
required for processing operation $(i,j)$ is denoted by $p_{ij}$. The operations
of a job need to be executed in a fixed given order, which may be different
among jobs, and an operation may only start once its predecessor has completed
processing. Each machine can process at most one operation at the same time and,
once started, operations cannot be preempted.

Let the set of all operations be denoted by $N$. Furthermore, we encode the job
routes by defining the set $A$ of precedence constraints
$(i,j) \xrightarrow{} (k,j)$. Let $y_{ij}$ denote the start of operation
$(i,j)$. The completion time of job $j$ is defined as
$C_{j} := y_{lj} + p_{lj}$, where $l$ is the last machine on which $j$ must be
processed. A valid schedule is given by setting values for $y_{ij}$ such that
the above requirements are met. There are various measures for how \textit{good}
a given schedule is. For the purpose of this example, let us consider the
well-known makespan objective $C_{\text{max}} := \max_{j} C_{j}$, which is often
related to efficient use of the available machines. Minimizing the makespan can
now be formulated as a Mixed-Integer Program (MIP) as follows:
%
\begin{align*}
  \text{minimize } C_{\text{max}} \\
  y_{ij} + p_{ij} &\leq y_{kj}  & \text{ for all } (i,j) \xrightarrow{} (k,j) \in A \\
  y_{il} + p_{il} &\leq  y_{ij} \text{ or } y_{ij} + p_{ij} \leq y_{il}  & \text{ for all } (i,l) \text{ and } (i,j), i =1, \dots,m \\
  y_{ij} + p_{ij} &\leq C_{\text{max}} & \text{ for all } (i,j) \in N \\
  y_{ij} &\geq 0 & \text{ for all } (i,j) \in N
\end{align*}
%
The first set of constraints enforce the order of operations belonging to the
same job. The second set of constraints are called \textit{disjunctive}, because
they model that we need to choose between jobs $l$ and $l$ to be scheduled first
on machine $i$. The next constraints are used to define the makespan and the
last line enforces non-negative start times.

\section{Traffic Scheduling}

% TODO: introduce multiple lanes, opposing lanes on the same arc and consider
% multiple phases at intersections
% TODO: use ``arc'' instead of ``arc''
We now turn to vehicles traveling through a network of intersections. The
network may be thought of as a weighted directed graph $G=(V,E)$, with nodes and
arcs representing intersections and roads, respectively. Let $d(x,y)$ be
defined as the \textit{distance} between nodes $x$ and $y$. We assume there are
no nodes of degree two, since their two incident arcs $(x,y)$ and $(y,z)$ could
be merged into one arc $(x,z)$ with $d(x,z) = d(x,y) + d(y,z)$, without loss of
expressiveness. Furthermore, we assume the graph is connected. Each node of
degree one is called an \textit{external node} and models the location where
vehicles enter/exit the network. A node of degree at least three is called an
\textit{internal node} and models an intersection.

Each vehicle $j$ enters the network at some external node $s$ and follows a
predetermined sequence of arcs
$R = ((s,i_{0}), (i_{0},i_{1}), \dots, (i_{n-1},i_{n}), (i_{n},d))$ towards an
external node $d$ where it leaves the network. Vehicles are not able to overtake
each other when traveling on the same arc. We assume that arcs provide
infinite \textit{buffers} for vehicles, meaning that there is no limit on the
number of vehicles that are traveling on the same arc at the same time.
However, we impose a minimum time required to travel along an arc $(x,y)$. By
assuming uniform maximum speed among vehicles, $d(x,y)$ can be directly
interpreted as this minimum travel time.

Let $y(i,j)$ denote the time vehicle $j$ enters intersection $i$. Crossing an
intersection takes $p$ time per vehicle and at most one vehicle can cross an
intersection at the same time. When two consecutive vehicles crossing an
intersection originate from the same arc, they may pass immediately after each
other. However, when a vehicle $j_{1}$ that wants to cross comes from a
different arc than the vehicle $j_{0}$ that last crossed the intersection, we
require that there is at least a \textit{switch-over time} $S$ between the
moment $j_{0}$ leaves and the moment $j_{1}$ enters the intersection.

Assuming that arrival times and routes of all vehicles are fixed and given,
our task is to determine when individual vehicles should cross intersections by
setting values for $y(i,j)$.
Like job shop, this problem may be formulated as a MIP.

\begin{align*}
  \text{maximize } P(y) \\
  y_{ij} + p_{ij} + t_{ik} &\leq y_{kj}  & \text{ for all } (i,j) \xrightarrow{} (k,j) \in A \\
  y_{ij} + p_{ij} &\leq C_{\text{max}} & \text{ for all } (i,j) \in N \\
  y_{il} + p_{il} &\leq  y_{ij} \text{ or } y_{ij} + p_{ij} \leq y_{il}  & \text{ for all } (i,l) \text{ and } (i,j), i =1, \dots,m \\
  y_{ij} &\geq 0 & \text{ for all } (i,j) \in N \\
  %
\end{align*}
where $P(y)$ denotes some unspecified performance metric, leaving for now the
question of what it means for a schedule to be \textit{good}.

Preserve ordering of vehicles on each incoming edge, because we assume vehicles do not overtake each other:
\begin{align*}
  r_{j} < r_{l} \implies (i,j) \rightarrow{} (i,l) \text{ for all } i = 1,\dots,m .
\end{align*}

\end{document}
